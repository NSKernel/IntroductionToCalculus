\chapter {微積溯源序}
\markright{微積溯源序}\gdef\firstsectiontitle{微積溯源序}
\thispagestyle{fancy}

  \uwave{微積溯源}八卷,前四卷爲微分術,後四卷爲積分術,乃算學中最深之事也。余既與西士\uline{傅蘭雅}譯畢\uwave{代數術}二十五卷,更思求其進境,故又與\uline{傅君}譯此書焉。先是咸豐年間,曾有\uline{海寗}\uline{李壬叔}與西士\uline{偉烈亞力}譯出\uwave{代微積拾級}一書,流播海内。余素與\uline{壬叔}相友,得讀其書,粗明微積二術之梗概。所以又譯此書者,蓋欲補其所略也。書中代數之式甚繁,校算不易,則\uline{劉君省菴}之力居多。

  今刻工已竣矣,故序之,曰:吾以爲古時之算法惟有加減而已。其乘與除乃因加減之不勝其繁,故更立二術以使之簡易也。開方之法,又所以濟除法之窮者也。蓋算學中自有加減乘除開方五法,而一切淺近易明之數,無不可通矣。惟人之心思智慮日出不窮,往往以能人之所不能者爲快。遇有窒礙難通之處,輒思立法以濟其窮。故有減其所不可減而正負之名不得不立矣;除其所不能除而寄母通分之法又不得不立矣。代數中種種記號之法皆出於不得已而立者也,惟每立一法必能使繁者爲簡,難者爲易,遲者爲速,而算學之境界藉此得更進一層。如是屢進不已而所立之法於是乎日多矣。微分積分者,蓋又因乘除開方之不勝其繁,且有窒礙難通之處,故更立此二術以濟其窮,又使簡易而速者也。試觀圓徑求周、真數求對等事,雖無微分積分之時,亦未嘗不可求,惟須乘除開方數十百次。其難有不可言喻者,不如用微積之法理明而數捷也。然則謂加減乘除開方代數之外者,更有二術焉,一曰微分,一曰積分可也。其積分術爲微分之還原,猶之開平方爲自乘之還原、除法爲乘之還原、減法爲加之還原也。然加與乘其原無不可還,而微分之原有可還有不可還,是猶算式中有不可開之方耳,又何怪焉。如必曰加減乘除開方已足供吾之用矣,何必更究其精?是舍舟車之便利而必欲負重遠行也。其用力多而成功少,蓋不待智者而辨矣。同治十三年九月十八日,\uline{金匱}\uline{華蘅芳}序。