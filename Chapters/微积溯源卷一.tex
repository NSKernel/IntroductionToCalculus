\chapter {微積溯源卷一}
\setcounter{page}{1}
\section {論變數与函數之變比例}
\thispagestyle{fancy}
\begin{enumerate} [label={第\chinese*款}]
	\item 用代數以解任何曲線,其中每有幾種數,其大小恒有定率者,如橢圓之長徑、抛物線之通徑、雙曲線之屬徑之類是也。\\
	又每有幾種數可有任若干相配之同數,其大小恒不能有定率者,如曲線任一點之縱橫線是也。\\
	數既有此兩種分別,則每種須有一總名以賅之,故名其有定之數曰常數,無定之數曰變數。\\
	凡常數之同數不能增亦不能損。\\
	凡變數之同數,能變爲大,亦能變爲小。故其從此同數變至彼同數之時,必歷彼此二數閒最小最微之各分數。\\
	如平圓之半徑爲常數,而其任一段之弧或弧之弦矢切割各線、及各線與弧所成之面,皆謂之變數。\\
	橢圓之長徑短徑皆爲常數,而其曲線之任一段或曲線上任一點之縱橫線,並其形内形外所能作之任何線或面或角,皆謂之變數。\\
	抛物線之通徑爲常數,而其曲線之任一段或任一點之縱橫線、或弧與縱橫線所成之面,皆謂之變數。他種曲線亦然。\\
	凡常數,恒以甲乙丙丁等字代之。凡變數,恒以天地人等字代之。
	\item 若有彼此二數皆爲變數,此數變而彼數因此數變而亦變者,則彼數爲此數之函數。\\
	如平圓之八線皆爲弧之函數。若反求之,亦可以弧爲八線之函數。\\
	又如重學中令物體前行之力,與其物所行之路,皆爲時刻之函數。\\
	如有式$\displaystyle \textit{地}=\frac{\textit{甲}\tang\textit{天}}{\textit{甲}\perp\textit{天}}$,此式中甲爲常數,天爲自主之變數,地爲天之函數。故地之同數能以天與甲明之。\\
	如有式$\displaystyle \textit{天}=\frac{\textit{地}\perp\textit{一}}{\textit{甲}(\textit{地}\tang\textit{一})}$,此式中甲與一皆爲常數,地爲自主之變數,天爲地之函數,故天之同數可以地與甲及一明之。\\
	如有式$\textit{戊}=\textit{甲}\perp\textit{乙天}\perp\textit{丙天}^{\textit{二}}$或$\displaystyle\textit{戊}=\sqrt{\textit{甲}^{\textit{二}}\perp\textit{乙天}\perp\textit{天}^{\textit{二}}}$或$\displaystyle\textit{戊}=\frac{\textit{丙}\perp\textit{天}^{\textit{二}}}{\textit{甲}\perp\sqrt{\textit{乙天}}}$,其甲乙丙爲常數,天爲自變之數,而戌皆爲天之函數。\\
	凡函數之中,可以有數箇自主之變數。\\
	
\end{enumerate}