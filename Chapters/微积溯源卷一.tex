\chapter {微積溯源卷一}
\setcounter{page}{1}
\section {論變數与函數之變比例}
\thispagestyle{fancy}
\begin{enumerate} [label={第\chinese*款}]
	\item 用代數以解任何曲線,其中每有幾種數,其大小恒有定率者,如橢圓之長徑、抛物線之通徑、雙曲線之屬徑之類是也。\\
	又每有幾種數可有任若干相配之同數,其大小恒不能有定率者,如曲線任一點之縱橫線是也。\\
	數既有此兩種分別,則每種須有一總名以賅之,故名其有定之數曰常數,無定之數曰變數。\\
	凡常數之同數不能增亦不能損。\\
	凡變數之同數,能變爲大,亦能變爲小。故其從此同數變至彼同數之時,必歷彼此二數閒最小最微之各分數。\\
	如平圓之半徑爲常數,而其任一段之弧或弧之弦矢切割各線、及各線與弧所成之面,皆謂之變數。\\
	橢圓之長徑短徑皆爲常數,而其曲線之任一段或曲線上任一點之縱橫線,並其形内形外所能作之任何線或面或角,皆謂之變數。\\
	抛物線之通徑爲常數,而其曲線之任一段或任一點之縱橫線、或弧與縱橫線所成之面,皆謂之變數。他種曲線亦然。\\
	凡常數,恒以甲乙丙丁等字代之。凡變數,恒以天地人等字代之。
	\item 若有彼此二數皆爲變數,此數變而彼數因此數變而亦變者,則彼數爲此數之函數。\\
	如平圓之八線皆爲弧之函數。若反求之,亦可以弧爲八線之函數。\\
	又如重學中令物體前行之力,與其物所行之路,皆爲時刻之函數。\\
	如有式\CJKmoveback{\equa{$\displaystyle \text{地}=\frac{\text{甲}\tang\text{天}}{\text{甲}\perp\text{天}}$}}\CJKmove,此式中甲爲常數,天爲自主之變數,地爲天之函數。故地之同數能以天與甲明之。\\
	如有式\CJKmoveback{\equa{$\displaystyle \textit{天}=\frac{\text{地}\perp\text{一}}{\text{甲}(\text{地}\tang\text{一})}$}}\CJKmove,此式中甲與一皆爲常數,地爲自主之變數,天爲地之函數,故天之同數可以地與甲及一明之。\\
	如有式\CJKmoveback{\equa{$\textit{戌}=\textit{甲}\perp\textit{乙天}\perp\textit{丙天}^{\textit{二}}$}}\CJKmove 或\CJKmoveback{\equa{$\displaystyle\textit{戌}=\sqrt{\textit{甲}^{\textit{二}}\perp\textit{乙天}\perp\textit{天}^{\textit{二}}}$}}\CJKmove 或\CJKmoveback{\equa{$\displaystyle\textit{戌}=\frac{\textit{丙}\perp\textit{天}^{\textit{二}}}{\textit{甲}\perp\sqrt{\textit{乙天}}}$}}\CJKmove,其甲乙丙爲常數,天爲自變之數,而戌皆爲天之函數。\\
	凡函數之中,可以有數箇自主之變數。\\
	如有式\CJKmoveback{\equa{$\textit{戌}=\textit{甲天}^{\textit{二}}\perp\textit{乙天地}\perp\textit{丙地}^{\textit{二}}$}}\CJKmove 則天與地皆爲自主之變數,戌爲天地兩變數之函數。\\
	凡變數之函數,其形雖有多種,然每可化之,使不外乎以下數類\CJKmoveback{\equa{$\textit{天}^{\text{卯}}$、$\textit{甲}^{\textit{天}}$、$\textit{正弦天}$、$\textit{餘弦天}$}}\CJKmove 等類是也。\\
	凡函數爲\CJKmoveback{\equa{$\textit{天}^{\text{卯}}$}}\CJKmove 之類,其指數爲常數,則可從天之卯方,用代數之常法化之。\\
	而以有窮之項明其函數之同數。故謂之代數函數,亦謂之常函數。\\
	如有式\CJKmoveback{\equa{$\displaystyle\textit{戌}=\textit{甲天}^{\textit{三}}\perp\frac{\sqrt{\textit{甲}^{\textit{二}}\perp\textit{天}^{\textit{二}}}}{\textit{乙天}^{\textit{二}}\tang\textit{丙}^{\textit{二}}\textit{天}}$}}\CJKmove,此種函數,其戌之同數可用加減乘除開方等法而得之。\\
	凡函數爲\CJKmoveback{\equa{甲、對天}}\CJKmove 之類,則其函數之同數不能以有窮之項明之。故謂之越函數。\textsuperscript{越者,超}\\\textsuperscript{越於尋常之意也。}\\
	凡函數爲\CJKmoveback{\equa{正弦天、餘弦天}}\CJKmove 及\CJKmoveback{\equa{正切天、正割天}}\CJKmove 之類,其函數之同數皆可以平圓之各線明之。故謂之圓函數,亦謂之角函數。\\
	以上三種函數\textsuperscript{常函數、越函數、圓函數也。},若已知天之同數,則其函數之同數即可求得。故名此三種函數爲陽函數。\textsuperscript{因其顯而易明,故謂之陽函數。}\\
	更有他種函數,必先解其方程式,令函數中之各變數分開,然後能求其同數者。\\
	如有式\CJKmoveback{\equa{$\displaystyle\textit{戌天}=\frac{\textit{戌}\tang\textit{天}}{\textit{戌}\perp\textit{天}}$}}\CJKmove,其戌爲天之函數,如欲求其戌與天相配之同數,必先解其二次方程式始能通。\\
	此種之式名曰天之陰函數。\textsuperscript{因其雜糅未明,故謂之陰函數。}反之亦可云天爲戌之陰函數。\\
	如解其方程式爲\CJKmoveback{\equa{$\displaystyle\textit{戌}=\frac{\textit{二}}{\textit{一}}\left[\textit{天}\perp\frac{\textit{天}}{\textit{一}}\perp\frac{\textit{天}}{\sqrt{\textit{天}^{\textit{四}}\perp\textit{六天}^{\textit{二}}\perp\textit{一}}}\right]$}}\CJKmove,則戌變爲天之陰函數。\\
	昔代數之家,凡遇須用開平方之處,每于其式之左旁作一根字以記之,如\CJKmoveback{\equa{根天}}\CJKmove 爲天之平方根。後又變通其法,而以根號記之,如\CJKmoveback{\equa{$\sqrt{\textit{天}}$}}\CJKmove 爲天之平方根。此代數之例也。茲可仿照此例,凡遇某變數之函數,亦用一號以記之。所以凡有任何變數之函數,皆可書一函字于其變數之旁,以爲識別。\\
	如天之函數則作\CJKmoveback{\equa{函天}}\CJKmove ,或作\CJKmoveback{\equa{函$(\textit{天})$}}\CJKmove ,皆言天之函數也。\\
	所以凡見變數之左旁有一函字者,其函字并非代表天之倍數,其意謂是某變數之函數也。\\
	用此法則可將\CJKmoveback{\equa{$\textit{戌}=\textit{天}^{\text{卯}}$、$\textit{戌}=\textit{甲}^{\textit{天}}$、$\textit{戌}=\textit{對天}$、$\textit{戌}=\textit{正弦天}$、$\textit{戌}=\textit{餘弦天}$}}\CJKmove 各種之式以一語賅之,謂之\CJKmoveback{\equa{$\textit{戌}=\textit{函天}$}}\CJKmove 或\CJKmoveback{\equa{$\textit{戌}=\textit{函}(\textit{天})$}}\CJKmove 。\\
	若函數從兩箇變數而成,其天與地皆爲自主之變數,其式如\CJKmoveback{\equa{$\textit{戌}=\textit{甲天}^{\textit{二}}\perp\textit{乙天地}\perp\textit{丙地}^{\textit{二}}$}}\CJKmove 者,則可以\CJKmoveback{\equa{$\textit{戌}=\textit{函}(\textit{天},\textit{地})$}}\CJKmove 別之。函數爲多箇變數所成者,仿此推之。\\
	惟函數只指其變數言之,若甲乙丙丁各常數。雖多不論。
	\item 凡觀此書者,必先明變數與函數變比例之限。如\uwave{幾何原本}中證明平圓之面積必比其外切多等邊形之面積微小,若其外切多等邊形之邊愈多,則其面積愈近于平圓之面積。所以可設平圓之面積爲任何小,而切其圓外爲多等邊形,可使多等邊形之面積與平圓之面積較其數甚小于所設之圓面積;再設其多等邊形之面積爲級數,而其邊之變率可每變多若干倍,則其多等邊形之面積必漸與平圓之面積相近而以平圓爲限。雖切于圓外之多邊形其邊任變至若何多,其面積總不能等于平圓之面積。然其級數之總數可比平圓之面積所差甚微,其較數之小,可小至莫可名言。\\
	若用此法于圓内容多等邊形,則其多等邊形之面積亦以平圓之面積爲限。\\
	總言之,凡平圓之周爲其内容外切多等邊形之限。\\
	如代數術第二百六十六款言,如令甲代平圓之任何弧,則\CJKmoveback{\equa{$\displaystyle\frac{\textit{正弦甲}}{\textit{甲}}$}}\CJKmove 恒小于半徑,而\CJKmoveback{\equa{$\displaystyle\frac{\textit{正切甲}}{\textit{甲}}$}}\CJKmove 恒小于半徑。然令其弧爲任何小,則其式之同數必甚近于半徑,而其所差之數可小至不能以言語形容。所以此兩數中間之數爲一。\textsuperscript{即半徑也。}故其公限亦爲一。\\
	由此可見凡弧與弦切,三者之中取其二以相較,其比例之限必相等。\\
	如代數術中亦會證甲弧爲\CJKmoveback{\equa{$\displaystyle\textit{卯正切}\frac{\textit{卯}}{\textit{甲}}$}}\CJKmove 與\CJKmoveback{\equa{$\displaystyle\textit{卯正弦}\frac{\textit{卯}}{\textit{甲}}$}}\CJKmove 兩式之限,惟其卯必爲任何大。\\
	依代數術第五十六款之例\CJKmoveback{\equa{$\textit{一}\perp\textit{天}\perp\textit{天}^\textit{二}\perp\textit{天}^\textit{三}\cdots\cdots\perp\textit{天}^{\textit{卯}\tang\textit{一}}$}}\CJKmove 即\CJKmoveback{\equa{$\displaystyle\frac{\textit{一}\tang \textit{天}}{\textit{一}\tang \textit{天}^\textit{卯}}$}}\CJKmove。若天變至小于一,而卯大至無窮,則\CJKmoveback{\equa{$\textit{天}^\textit{卯}=\textit{〇}$}}\CJKmove ,而式變爲\CJKmoveback{\equa{$\displaystyle\frac{\textit{一}\tang \textit{天}}{\textit{一}}$}}\CJKmove。所以任取其級數若干項之和,必小于\CJKmoveback{\equa{$\displaystyle\frac{\textit{一}\tang \textit{天}}{\textit{一}}$}}\CJKmove。惟其項愈多,則與\CJKmoveback{\equa{$\displaystyle\frac{\textit{一}\tang \textit{天}}{\textit{一}}$}}\CJKmove 愈相近,而其所差之數可小至莫可名言。則可見\CJKmoveback{\equa{$\displaystyle\frac{\textit{一}\tang \textit{天}}{\textit{一}}$}}\CJKmove 必爲其諸級數之限。\\
	若依二項例之式\CJKmoveback{\equa{$\displaystyle\left(\textit{一}\perp\frac{\textit{卯}}{\textit{天}}\right)^\textit{卯}=\textit{一}\perp\frac{\textit{一}}{\textit{天}}\perp\frac{\textit{一}\cdot\textit{二}}{\textit{天}^\textit{二}}\left(\textit{一}\perp\frac{\textit{卯}}{\textit{一}}\right)\perp\frac{\textit{一}\cdot\textit{二}\cdot\textit{三}}{\textit{天}^\textit{三}}\left(\textit{一}\perp\frac{\textit{卯}}{\textit{一}}\right)$\\$\displaystyle\left(\textit{一}\perp\frac{\textit{卯}}{\textit{二}}\right)\perp\cdots\cdots$}}\CJKmove ,其卯之同數無論如何,必合于理。惟卯若爲大數,則其各項之乘數\CJKmoveback{\equa{$\displaystyle\textit{一}\perp\frac{\textit{卯}}{\textit{一}}$、$\displaystyle\textit{一}\perp\frac{\textit{卯}}{\textit{二}}$}}\CJKmove 之類與一相較之差甚小,若卯愈大,則其差愈微;若令卯爲任意大,則各乘數可略等于一,所以得\CJKmoveback{\equa{$\displaystyle\left(\textit{一}\perp\frac{\textit{卯}}{\textit{天}}\right)^\textit{卯}=\textit{一}\perp\frac{\textit{一}}{\textit{天}}\perp\frac{\textit{一}\cdot\textit{二}}{\textit{天}^\textit{二}}\perp\frac{\textit{一}\cdot\textit{二}\cdot\textit{三}}{\textit{天}^\textit{三}}\perp\cdots\cdots$}}\CJKmove \raisebox{1pt} {\textcircled{\raisebox{-.5pt} {\small\textit{甲}}}}。\\
	曾在代數術第一百七十七款中證\raisebox{1pt} {\textcircled{\raisebox{-.5pt} {\small\textit{甲}}}}式之右邊爲由函數\CJKmoveback{\equa{$\textit{戊}^\textit{天}$}}\CJKmove 而成。其戊之同數因爲\CJKmoveback{\equa{二.七一八二八一八}}\CJKmove,即\uline{訥白爾}對數之根也。所以卯若愈大則\CJKmoveback{\equa{$\displaystyle\left(\textit{一}\perp\frac{\textit{卯}}{\textit{天}}\right)^\textit{卯}$}}\CJKmove 必愈與\CJKmoveback{\equa{$\textit{戊}^\textit{天}$}}\CJKmove 相近,而其限爲\CJKmoveback{\equa{$\textit{戊}^\textit{天}$}}\CJKmove。\\
	如令\CJKmoveback{\equa{$\textit{天}=\textit{一}$}},則\CJKmoveback{\equa{$\displaystyle\left(\textit{一}\perp\frac{\textit{卯}}{\textit{天}}\right)^\textit{卯}$}}\CJKmove 之限爲戊,即\CJKmoveback{\equa{$\displaystyle\left(\textit{一}\perp\frac{\textit{卯}}{\textit{天}}\right)^\textit{卯}-\textit{戊}=\text{二.七一八二八一八}$}}\CJKmove,故其函數爲常數。
	\item 惟因函數之同數本從變數而生,故變數之同數與函數之長數比則爲\CJKmoveback{\equa{$\displaystyle\frac{\textit{辛}}{\textit{戌}'\tang\textit{戌}}=\textit{二天}\perp\textit{辛}$}}\CJKmove。\\
	設函數之式爲\CJKmoveback{\equa{$\textit{戌}=\textit{天}^\textit{三}$}}\CJKmove,令天長數爲辛,而以函數之新同數爲\CJKmoveback{\equa{$\textit{戌}'$}}\CJKmove 則\\
	\CJKmoveback{\equa{
			$\displaystyle\textit{戌}'=(\textit{天}\perp\textit{辛})^\textit{三}\begin{cases}
	            =\textit{天}^\textit{三}\perp\textit{三天}^\textit{二}\textit{辛}\perp\textit{三天辛}^\textit{二}\perp\textit{辛}^\textit{三}\\
	            =\textit{戌}\perp\textit{三天}^\textit{二}\textit{辛}\perp\textit{三天辛}^\textit{二}\perp\textit{辛}^\textit{三}
	        \end{cases}$
	}}\CJKmove,而\CJKmoveback{\equa{$\displaystyle\frac{\textit{辛}}{\textit{戌}'\tang\textit{戌}}=\textit{三天}^\textit{二}\perp\textit{三天辛}\perp\textit{辛}^\textit{二}$}}\CJKmove。可見天變爲\CJKmoveback{\equa{$\textit{天}\perp\textit{辛}$}}\CJKmove 之時,其函數\CJKmoveback{\equa{$\textit{戌}$}}\CJKmove 必變爲\CJKmoveback{\equa{$\textit{戌}\perp\textit{三天}^\textit{二}\textit{辛}\perp\textit{三天辛}^\textit{二}\perp\textit{辛}^\textit{三}$}}\CJKmove,其所長之數爲\CJKmoveback{\equa{$\textit{三天}^\textit{二}\textit{辛}\perp\textit{三天辛}^\textit{二}\perp\textit{辛}^\textit{三}$}}\CJKmove。此式中之各項皆爲辛之整方與他數相乘所成。又可見變數與函數之變比例,其式爲\CJKmoveback{\equa{$\textit{三天}^\textit{二}\textit{辛}\perp\textit{三天辛}^\textit{二}\perp\textit{辛}^\textit{三}$}}\CJKmove,其初項\CJKmoveback{\equa{$\textit{三天}^\textit{二}$}}\CJKmove 與天之長數辛無相關。\\
    設函數之式爲\CJKmoveback{\equa{$\textit{戌}=\textit{天}^\textit{四}$}}\CJKmove。令天之長數爲辛,而以\CJKmoveback{\equa{$\textit{戌}'$}}\CJKmove 爲函數之新同數,則\CJKmoveback{\equa{$\textit{戌}'=\textit{戌}\perp\textit{四天}^\textit{三}\textit{辛}\perp\textit{六天}^\textit{二}\textit{辛}^\textit{二}\perp\textit{四天辛}^\textit{三}\perp\textit{辛}^\textit{四}$}}\CJKmove,而\CJKmoveback{\equa{$\displaystyle\frac{\textit{辛}}{\textit{戌}'\tang\textit{戌}}=\textit{四天}^\textit{三}\perp\textit{六天}^\textit{二}\textit{辛}\perp\textit{四天辛}^\textit{二}\perp\textit{辛}^\textit{三}$}}\CJKmove。\\
    由此可見天若變爲\CJKmoveback{\equa{$\textit{天}\perp\textit{辛}$}}\CJKmove,則其各函數之新同數如左:\\
    如\CJKmoveback{\equa{$\textit{戌}=\textit{天}^\textit{二}$}}\CJKmove,則\CJKmoveback{\equa{$\textit{戌}'=\textit{戌}\perp\textit{二天辛}\perp\textit{辛}^\textit{二}$}}\CJKmove。如\CJKmoveback{\equa{$\textit{戌}=\textit{天}^\textit{三}$}}\CJKmove,則\CJKmoveback{\equa{$\textit{戌}'=\textit{戌}\perp\textit{三天}^\textit{二}\textit{辛}\perp\textit{三天辛}^\textit{二}\perp\textit{辛}^\textit{三}$}}\CJKmove。如\CJKmoveback{\equa{$\textit{戌}=\textit{天}^\textit{四}$}}\CJKmove,則\CJKmoveback{\equa{$\textit{戌}'=\textit{戌}\perp\textit{四天}^\textit{三}\textit{辛}\perp\textit{六天}^\textit{二}\textit{辛}^\textit{二}\perp\textit{四天辛}^\textit{三}\perp\textit{辛}^\textit{四}$}}\CJKmove。其餘類推。\\
    總言之,若以卯爲天之任何整指數,而令天之長數爲辛,又以巳午未申等字挨次而代辛之各方之倍數,則函數\CJKmoveback{\equa{$\textit{戌}=\textit{天}^\textit{卯}$}}\CJKmove 之新同數必爲\CJKmoveback{\equa{$\textit{戌}'=\textit{辛}\perp\textit{巳辛}\perp\textit{午辛}^\textit{二}\perp\textit{未辛}^\textit{三}\perp\textit{申辛}^\textit{四}\perp\cdots\cdots$}}\CJKmove。由是知函數之新同數必爲級數,其初項戌爲函數之原同數,其餘各項爲天之長數辛之各整方,以巳午未申之類爲各倍數,其各倍數皆爲天之別種函數,其式亦從本函數而生。\\
    由以上各式又可見函數爲\CJKmoveback{\equa{$\textit{戌}=\textit{天}^\textit{二}$、$\textit{戌}=\textit{天}^\textit{三}$、$\textit{戌}=\textit{天}^\textit{四}$}}\CJKmove 之類,則其變數與函數之變比例必爲\CJKmoveback{\equa{$\displaystyle\frac{\textit{辛}}{\textit{戌}'\tang\textit{戌}}=\textit{二天}\perp\textit{辛}$、$\displaystyle\frac{\textit{辛}}{\textit{戌}'\tang\textit{戌}}=\textit{三天}^\textit{二}\perp\textit{三天辛}\perp\textit{辛}^\textit{二}$、$\displaystyle\frac{\textit{辛}}{\textit{戌}'\tang\textit{戌}}=\textit{四天}^\textit{三}\perp\textit{六天}^\textit{二}\textit{辛}\perp\textit{四天辛}^\textit{二}\perp\textit{辛}^\textit{三}$}}\CJKmove 之類。總之若以卯爲天之整指數,則\CJKmoveback{\equa{$\textit{戌}=\textit{天}^\textit{卯}$}}\CJKmove 之變比例必爲\CJKmoveback{\equa{$\displaystyle\frac{\textit{辛}}{\textit{戌}'\tang\textit{戌}}=\textit{巳}\perp\textit{午辛}\perp\textit{未辛}^\textit{二}\perp\textit{申辛}^\textit{三}\perp\cdots\cdots$}}\CJKmove。由此可見,變數天之長數與函數\CJKmoveback{\equa{$\textit{天}^\textit{卯}$}}\CJKmove 之長數其變比例\CJKmoveback{\equa{$\displaystyle\frac{\textit{辛}}{\textit{戌}'\tang\textit{戌}}$}}\CJKmove 之同數\CJKmoveback{\equa{$\textit{巳}\perp\textit{午辛}\perp\textit{未辛}^\textit{二}\perp\textit{申辛}^\textit{三}\perp\cdots\cdots$}}\CJKmove 可分之爲兩式,其一式爲\CJKmoveback{\equa{$\textit{巳}$}}\CJKmove,此式與天之長數辛無相關;又一式爲\CJKmoveback{\equa{$\textit{午辛}\perp\textit{未辛}^\textit{二}\perp\textit{申辛}^\textit{三}\perp\cdots\cdots$}}\CJKmove。即\CJKmoveback{\equa{$\textit{辛}\left(\textit{午}\perp\textit{未辛}\perp\textit{申辛}^\textit{二}\perp\cdots\cdots\right)$}}\CJKmove。此式因以辛爲乘數,故辛若變小,其數亦必隨辛而變小。如令辛爲任何小,則此式之數可小至甚近于〇。故此數可以不計,而以巳爲變比例之限。\\
    設有繁函數之式\CJKmoveback{\equa{$\textit{戌}=\textit{甲}\perp\textit{乙天}\perp\textit{丙天}^\textit{二}$}}\CJKmove,令天之長數爲辛,則天變爲\CJKmoveback{\equa{$\textit{天}\perp\textit{辛}$}}\CJKmove 之時,其函數之同數必變爲\CJKmoveback{\equa{$\displaystyle\textit{戌}\begin{cases}
        =\textit{甲}\perp\textit{乙}(\textit{天}\perp\textit{辛})\perp\textit{丙}(\textit{天}\perp\textit{辛})^\textit{二}\\
        =\textit{甲}\perp\textit{乙天}\perp\textit{丙天}^\textit{二}\perp(\textit{乙}\perp\textit{二丙天})\textit{辛}\perp\textit{丙辛}^\textit{二}
    \end{cases}$}}\CJKmove,故\\\CJKmoveback{\equa{$\textit{戌}'\tang\textit{戌}=(\textit{乙}\perp\textit{二丙天})\textit{辛}\perp\textit{丙辛}^\textit{二}$}}\CJKmove,而\CJKmoveback{\equa{$\displaystyle\frac{\textit{辛}}{\textit{戌}'\tang\textit{戌}}=\textit{乙}\perp\textit{二丙天}\perp\textit{丙辛}$}}\CJKmove。其\CJKmoveback{\equa{$\textit{乙}\perp\textit{二丙天}$}}\CJKmove 爲本函數變比例之限。\\
    以此法徧試各種特設之函數,見其皆有相類之性情,所以例設如左。\\
    例曰:命任何自主之變數爲天,而令天之任何函數等于戌,則天變爲\CJKmoveback{\equa{$\textit{天}\perp\textit{辛}$}}\CJKmove 之時,函數之新同數爲\CJKmoveback{\equa{$\textit{戌}'=\textit{戌}\perp\textit{巳辛}\perp\textit{午辛}^\textit{二}\perp\textit{未辛}^\textit{三}\perp\cdots\cdots$}}\CJKmove,其變數與函數之變比例爲\CJKmoveback{\equa{$\displaystyle\frac{\textit{辛}}{\textit{戌}'\tang\textit{戌}}=\textit{巳}\perp\textit{午辛}\perp\textit{未辛}^\textit{二}\perp\textit{申辛}^\textit{三}\perp\cdots\cdots$}}\CJKmove。此式中之初項\CJKmoveback{\equa{$\textit{巳}$}}\CJKmove 爲變比例之限,無論何種函數,其限皆可依此比例求之。\\
    由以上所論變數之長數與函數之長數相關之理,可于算學中開出兩種極廣大極精微之法。\\
    其第一種爲有任何變數之任何函數而求其變數與函數變比例之限。\\
    其第二種爲有任何變數與函數變比例之限而求其函數之原式。\\
    此二種法,若細攷其根源,即\uline{奈端}所謂正流數、反流數也;亦即\uline{來本之}所謂微分算術、積分算術也;又即\uline{拉果闌諸}所謂函數變例也。
\end{enumerate}

