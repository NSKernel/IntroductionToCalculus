\chapter {微積溯源卷一}
\setcounter{page}{1}
\section {論變數与函數之變比例}
\thispagestyle{fancy}
\begin{enumerate} [label={第\chinese*款}]
	\item 用代數以解任何曲線,其中每有幾種數,其大小恒有定率者,如橢圓之長徑、抛物線之通徑、雙曲線之屬徑之類是也。\\
	又每有幾種數可有任若干相配之同數,其大小恒不能有定率者,如曲線任一點之縱橫線是也。\\
	數既有此兩種分別,則每種須有一總名以賅之,故名其有定之數曰常數,無定之數曰變數。\\
	凡常數之同數不能增亦不能損。\\
	凡變數之同數,能變爲大,亦能變爲小。故其從此同數變至彼同數之時,必歷彼此二數閒最小最微之各分數。\\
	如平圓之半徑爲常數,而其任一段之弧或弧之弦矢切割各線、及各線與弧所成之面,皆謂之變數。\\
	橢圓之長徑短徑皆爲常數,而其曲線之任一段或曲線上任一點之縱橫線,並其形内形外所能作之任何線或面或角,皆謂之變數。\\
	抛物線之通徑爲常數,而其曲線之任一段或任一點之縱橫線、或弧與縱橫線所成之面,皆謂之變數。他種曲線亦然。\\
	凡常數,恒以甲乙丙丁等字代之。凡變數,恒以天地人等字代之。
	\item 若有彼此二數皆爲變數,此數變而彼數因此數變而亦變者,則彼數爲此數之函數。\\
	如平圓之八線皆爲弧之函數。若反求之,亦可以弧爲八線之函數。\\
	又如重學中令物體前行之力,與其物所行之路,皆爲時刻之函數。\\
	如有式\CJKmoveback{\equa{$\displaystyle \text{地}=\frac{\text{甲}\tang\text{天}}{\text{甲}\perp\text{天}}$}}\CJKmove,此式中甲爲常數,天爲自主之變數,地爲天之函數。故地之同數能以天與甲明之。\\
	如有式\CJKmoveback{\equa{$\displaystyle \textit{天}=\frac{\text{地}\perp\text{一}}{\text{甲}(\text{地}\tang\text{一})}$}}\CJKmove,此式中甲與一皆爲常數,地爲自主之變數,天爲地之函數,故天之同數可以地與甲及一明之。\\
	如有式\CJKmoveback{\equa{$\textit{戌}=\textit{甲}\perp\textit{乙天}\perp\textit{丙天}^{\textit{二}}$}}\CJKmove 或\CJKmoveback{\equa{$\displaystyle\textit{戌}=\sqrt{\textit{甲}^{\textit{二}}\perp\textit{乙天}\perp\textit{天}^{\textit{二}}}$}}\CJKmove 或\CJKmoveback{\equa{$\displaystyle\textit{戌}=\frac{\textit{丙}\perp\textit{天}^{\textit{二}}}{\textit{甲}\perp\sqrt{\textit{乙天}}}$}}\CJKmove,其甲乙丙爲常數,天爲自變之數,而戌皆爲天之函數。\\
	凡函數之中,可以有數箇自主之變數。\\
	如有式\CJKmoveback{\equa{$\textit{戌}=\textit{甲天}^{\textit{二}}\perp\textit{乙天地}\perp\textit{丙地}^{\textit{二}}$}}\CJKmove 則天與地皆爲自主之變數,戌爲天地兩變數之函數。\\
	凡變數之函數,其形雖有多種,然每可化之,使不外乎以下數類\CJKmoveback{\equa{$\textit{天}^{\text{卯}}$}}\CJKmove、\CJKmoveback{\equa{$\textit{甲}^{\textit{天}}$、$\textit{正弦天}$、$\textit{餘弦天}$}}\CJKmove 等類是也。\\
	凡函數爲\CJKmoveback{\equa{$\textit{天}^{\text{卯}}$}}\CJKmove 之類,其指數爲常數,則可從天之卯方,用代數之常法化之。\\
	而以有窮之項明其函數之同數。故謂之代數函數,亦謂之常函數。\\
\end{enumerate}