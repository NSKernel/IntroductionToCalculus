\chapter {微積溯源卷一}
\setcounter{page}{1}
\section {論變數與函數之變比例}
\thispagestyle{fancy}
\begin{enumerate} [label={第\chinese*款}]
	\item 用代數以解任何曲線,其中每有幾種數,其大小恒有定率者,如橢圓之長徑、抛物線之通徑、雙曲線之屬徑之類是也。\\
	又每有幾種數可有任若干相配之同數,其大小恒不能有定率者,如曲線任一點之縱橫線是也。\\
	數既有此兩種分別,則每種須有一總名以賅之,故名其有定之數曰常數,無定之數曰變數。\\
	凡常數之同數不能增亦不能損。\\
	凡變數之同數,能變爲大,亦能變爲小。故其從此同數變至彼同數之時,必歷彼此二數閒最小最微之各分數。\\
	如平圓之半徑爲常數,而其任一段之弧或弧之弦矢切割各線、及各線與弧所成之面,皆謂之變數。\\
	橢圓之長徑短徑皆爲常數,而其曲線之任一段或曲線上任一點之縱橫線,並其形内形外所能作之任何線或面或角,皆謂之變數。\\
	抛物線之通徑爲常數,而其曲線之任一段或任一點之縱橫線、或弧與縱橫線所成之面,皆謂之變數。他種曲線亦然。\\
	凡常數,恒以甲乙丙丁等字代之。凡變數,恒以天地人等字代之。
	\item 若有彼此二數皆爲變數,此數變而彼數因此數變而亦變者,則彼數爲此數之函數。\\
	如平圓之八線皆爲弧之函數。若反求之,亦可以弧爲八線之函數。\\
	又如重學中令物體前行之力,與其物所行之路,皆爲時刻之函數。\\
	如有式\CJKmoveback{\equa{$\displaystyle \text{地}=\frac{\text{甲}\MinusSign\text{天}}{\text{甲}\PlusSign\text{天}}$}}\CJKmove,此式中甲爲常數,天爲自主之變數,地爲天之函數。故地之同數能以天與甲明之。\\
	如有式\CJKmoveback{\equa{$\displaystyle \textit{天}=\frac{\text{地}\PlusSign\text{一}}{\text{甲}(\text{地}\MinusSign\text{一})}$}}\CJKmove,此式中甲與一皆爲常數,地爲自主之變數,天爲地之函數,故天之同數可以地與甲及一明之。\\
	如有式\CJKmoveback{\equa{$\textit{戌}=\textit{甲}\PlusSign\textit{乙天}\PlusSign\textit{丙天}^{\textit{二}}$}}\CJKmove 或\CJKmoveback{\equa{$\displaystyle\textit{戌}=\sqrt{\textit{甲}^{\textit{二}}\PlusSign\textit{乙天}\PlusSign\textit{天}^{\textit{二}}}$}}\CJKmove 或\CJKmoveback{\equa{$\displaystyle\textit{戌}=\frac{\textit{丙}\PlusSign\textit{天}^{\textit{二}}}{\textit{甲}\PlusSign\sqrt{\textit{乙天}}}$}}\CJKmove,其甲乙丙爲常數,天爲自變之數,而戌皆爲天之函數。\\
	凡函數之中,可以有數箇自主之變數。\\
	如有式\CJKmoveback{\equa{$\textit{戌}=\textit{甲天}^{\textit{二}}\PlusSign\textit{乙天地}\PlusSign\textit{丙地}^{\textit{二}}$}}\CJKmove 則天與地皆爲自主之變數,戌爲天地兩變數之函數。\\
	凡變數之函數,其形雖有多種,然每可化之,使不外乎以下數類\CJKmoveback{\equa{$\textit{天}^{\text{卯}}$、$\textit{甲}^{\textit{天}}$、$\textit{正弦天}$、$\textit{餘弦天}$}}\CJKmove 等類是也。\\
	凡函數爲\CJKmoveback{\equa{$\textit{天}^{\text{卯}}$}}\CJKmove 之類,其指數爲常數,則可從天之卯方,用代數之常法化之。\\
	而以有窮之項明其函數之同數。故謂之代數函數,亦謂之常函數。\\
	如有式\CJKmoveback{\equa{$\displaystyle\textit{戌}=\textit{甲天}^{\textit{三}}\PlusSign\frac{\sqrt{\textit{甲}^{\textit{二}}\PlusSign\textit{天}^{\textit{二}}}}{\textit{乙天}^{\textit{二}}\MinusSign\textit{丙}^{\textit{二}}\textit{天}}$}}\CJKmove,此種函數,其戌之同數可用加減乘除開方等法而得之。\\
	凡函數爲\CJKmoveback{\equa{甲、對天}}\CJKmove 之類,則其函數之同數不能以有窮之項明之。故謂之越函數。\textsuperscript{越者,超}\\\textsuperscript{越於尋常之意也。}\\
	凡函數爲\CJKmoveback{\equa{正弦天、餘弦天}}\CJKmove 及\CJKmoveback{\equa{正切天、正割天}}\CJKmove 之類,其函數之同數皆可以平圓之各線明之。故謂之圓函數,亦謂之角函數。\\
	以上三種函數\textsuperscript{常函數、越函數、圓函數也。},若已知天之同數,則其函數之同數卽可求得。故名此三種函數爲陽函數。\textsuperscript{因其顯而易明,故謂之陽函數。}\\
	更有他種函數,必先解其方程式,令函數中之各變數分開,然後能求其同數者。\\
	如有式\CJKmoveback{\equa{$\displaystyle\textit{戌天}=\frac{\textit{戌}\MinusSign\textit{天}}{\textit{戌}\PlusSign\textit{天}}$}}\CJKmove,其戌爲天之函數,如欲求其戌與天相配之同數,必先解其二次方程式始能通。\\
	此種之式名曰天之陰函數。\textsuperscript{因其雜糅未明,故謂之陰函數。}反之亦可云天爲戌之陰函數。\\
	如解其方程式爲\CJKmoveback{\equa{$\displaystyle\textit{戌}=\frac{\textit{二}}{\textit{一}}\left[\textit{天}\PlusSign\frac{\textit{天}}{\textit{一}}\PlusSign\frac{\textit{天}}{\sqrt{\textit{天}^{\textit{四}}\PlusSign\textit{六天}^{\textit{二}}\PlusSign\textit{一}}}\right]$}}\CJKmove,則戌變爲天之陰函數。\\
	\textsuperscript{\textbf{整理者註:}右最後一句云\textit{戌變爲天之陰函數},這顯而易見是錯誤的。原文\uwave{大英百科全書}卷第九之Fluxions作explicit function,}\\\textsuperscript{卽陽函數,此當爲譯者譯錯。目前由於校勘凡例未定,暫留此字,待討論後決定修改與否。}\\
	昔代數之家,凡遇須用開平方之處,每于其式之左旁作一根字以記之,如\CJKmoveback{\equa{根天}}\CJKmove 爲天之平方根。後又變通其法,而以根號記之,如\CJKmoveback{\equa{$\sqrt{\textit{天}}$}}\CJKmove 爲天之平方根。此代數之例也。茲可仿照此例,凡遇某變數之函數,亦用一號以記之。所以凡有任何變數之函數,皆可書一函字于其變數之旁,以爲識別。\\
	如天之函數則作\CJKmoveback{\equa{函天}}\CJKmove ,或作\CJKmoveback{\equa{函$(\textit{天})$}}\CJKmove ,皆言天之函數也。\\
	所以凡見變數之左旁有一函字者,其函字并非代表天之倍數,其意謂是某變數之函數也。\\
	用此法則可將\CJKmoveback{\equa{$\textit{戌}=\textit{天}^{\text{卯}}$、$\textit{戌}=\textit{甲}^{\textit{天}}$、$\textit{戌}=\textit{對天}$、$\textit{戌}=\textit{正弦天}$、$\textit{戌}=\textit{餘弦天}$}}\CJKmove 各種之式以一語賅之,謂之\CJKmoveback{\equa{$\textit{戌}=\textit{函天}$}}\CJKmove 或\CJKmoveback{\equa{$\textit{戌}=\textit{函}(\textit{天})$}}\CJKmove 。\\
	若函數從兩箇變數而成,其天與地皆爲自主之變數,其式如\CJKmoveback{\equa{$\textit{戌}=\textit{甲天}^{\textit{二}}\PlusSign\textit{乙天地}\PlusSign\textit{丙地}^{\textit{二}}$}}\CJKmove 者,則可以\CJKmoveback{\equa{$\textit{戌}=\textit{函}(\textit{天},\textit{地})$}}\CJKmove 別之。函數爲多箇變數所成者,仿此推之。\\
	惟函數只指其變數言之,若甲乙丙丁各常數。雖多不論。
	\item 凡觀此書者,必先明變數與函數變比例之限。如\uwave{幾何原本}中證明平圓之面積必比其外切多等邊形之面積微小,若其外切多等邊形之邊愈多,則其面積愈近于平圓之面積。所以可設平圓之面積爲任何小,而切其圓外爲多等邊形,可使多等邊形之面積與平圓之面積較其數甚小于所設之圓面積;再設其多等邊形之面積爲級數,而其邊之變率可每變多若干倍,則其多等邊形之面積必漸與平圓之面積相近而以平圓爲限。雖切于圓外之多邊形其邊任變至若何多,其面積總不能等于平圓之面積。然其級數之總數可比平圓之面積所差甚微,其較數之小,可小至莫可名言。\\
	若用此法于圓内容多等邊形,則其多等邊形之面積亦以平圓之面積爲限。\\
	總言之,凡平圓之周爲其内容外切多等邊形之限。\\
	如\uwave{代數術}第二百六十六款言,如令甲代平圓之任何弧,則\CJKmoveback{\equa{$\displaystyle\frac{\textit{正弦甲}}{\textit{甲}}$}}\CJKmove 恒小于半徑,而\CJKmoveback{\equa{$\displaystyle\frac{\textit{正切甲}}{\textit{甲}}$}}\CJKmove 恒小于半徑。然令其弧爲任何小,則其式之同數必甚近于半徑,而其所差之數可小至不能以言語形容。所以此兩數中間之數爲一。\textsuperscript{卽半徑也。}故其公限亦爲一。\\
	由此可見凡弧與弦切,三者之中取其二以相較,其比例之限必相等。\\
	如\uwave{代數術}中亦會證甲弧爲\CJKmoveback{\equa{$\displaystyle\textit{卯正切}\frac{\textit{卯}}{\textit{甲}}$}}\CJKmove 與\CJKmoveback{\equa{$\displaystyle\textit{卯正弦}\frac{\textit{卯}}{\textit{甲}}$}}\CJKmove 兩式之限,惟其卯必爲任何大。\\
	依\uwave{代數術}第五十六款之例\CJKmoveback{\equa{$\textit{一}\PlusSign\textit{天}\PlusSign\textit{天}^\textit{二}\PlusSign\textit{天}^\textit{三}\cdots\cdots\PlusSign\textit{天}^{\textit{卯}\MinusSign\textit{一}}$}}\CJKmove 卽\CJKmoveback{\equa{$\displaystyle\frac{\textit{一}\MinusSign \textit{天}}{\textit{一}\MinusSign \textit{天}^\textit{卯}}$}}\CJKmove。若天變至小于一,而卯大至無窮,則\CJKmoveback{\equa{$\textit{天}^\textit{卯}=\textit{〇}$}}\CJKmove ,而式變爲\CJKmoveback{\equa{$\displaystyle\frac{\textit{一}\MinusSign \textit{天}}{\textit{一}}$}}\CJKmove。所以任取其級數若干項之和,必小于\CJKmoveback{\equa{$\displaystyle\frac{\textit{一}\MinusSign \textit{天}}{\textit{一}}$}}\CJKmove。惟其項愈多,則與\CJKmoveback{\equa{$\displaystyle\frac{\textit{一}\MinusSign \textit{天}}{\textit{一}}$}}\CJKmove 愈相近,而其所差之數可小至莫可名言。則可見\CJKmoveback{\equa{$\displaystyle\frac{\textit{一}\MinusSign \textit{天}}{\textit{一}}$}}\CJKmove 必爲其諸級數之限。\\
	若依二項例之式\CJKmoveback{\equa{$\displaystyle\left(\textit{一}\PlusSign\frac{\textit{卯}}{\textit{天}}\right)^\textit{卯}=\textit{一}\PlusSign\frac{\textit{一}}{\textit{天}}\PlusSign\frac{\textit{一}\cdot\textit{二}}{\textit{天}^\textit{二}}\left(\textit{一}\PlusSign\frac{\textit{卯}}{\textit{一}}\right)\PlusSign\frac{\textit{一}\cdot\textit{二}\cdot\textit{三}}{\textit{天}^\textit{三}}\left(\textit{一}\PlusSign\frac{\textit{卯}}{\textit{一}}\right)$\\$\displaystyle\left(\textit{一}\PlusSign\frac{\textit{卯}}{\textit{二}}\right)\PlusSign\cdots\cdots$}}\CJKmove ,其卯之同數無論如何,必合于理。惟卯若爲大數,則其各項之乘數\CJKmoveback{\equa{$\displaystyle\textit{一}\PlusSign\frac{\textit{卯}}{\textit{一}}$、$\displaystyle\textit{一}\PlusSign\frac{\textit{卯}}{\textit{二}}$}}\CJKmove 之類與一相較之差甚小,若卯愈大,則其差愈微;若令卯爲任意大,則各乘數可略等于一,所以得\CJKmoveback{\equa{$\displaystyle\left(\textit{一}\PlusSign\frac{\textit{卯}}{\textit{天}}\right)^\textit{卯}=\textit{一}\PlusSign\frac{\textit{一}}{\textit{天}}\PlusSign\frac{\textit{一}\cdot\textit{二}}{\textit{天}^\textit{二}}\PlusSign\frac{\textit{一}\cdot\textit{二}\cdot\textit{三}}{\textit{天}^\textit{三}}\PlusSign\cdots\cdots$}}\CJKmove \raisebox{1pt} {\textcircled{\raisebox{-.5pt} {\small\textit{甲}}}}。\\
	曾在\uwave{代數術}第一百七十七款中證\raisebox{1pt} {\textcircled{\raisebox{-.5pt} {\small\textit{甲}}}}式之右邊爲由函數\CJKmoveback{\equa{$\textit{戊}^\textit{天}$}}\CJKmove 而成。其戊之同數因爲\CJKmoveback{\equa{二.七一八二八一八}}\CJKmove,卽\uline{訥白爾}對數之根也。所以卯若愈大則\CJKmoveback{\equa{$\displaystyle\left(\textit{一}\PlusSign\frac{\textit{卯}}{\textit{天}}\right)^\textit{卯}$}}\CJKmove 必愈與\CJKmoveback{\equa{$\textit{戊}^\textit{天}$}}\CJKmove 相近,而其限爲\CJKmoveback{\equa{$\textit{戊}^\textit{天}$}}\CJKmove。\\
	如令\CJKmoveback{\equa{$\textit{天}=\textit{一}$}},則\CJKmoveback{\equa{$\displaystyle\left(\textit{一}\PlusSign\frac{\textit{卯}}{\textit{天}}\right)^\textit{卯}$}}\CJKmove 之限爲戊,卽\CJKmoveback{\equa{$\displaystyle\left(\textit{一}\PlusSign\frac{\textit{卯}}{\textit{天}}\right)^\textit{卯}-\textit{戊}=\text{二.七一八二八一八}$}}\CJKmove,故其函數爲常數。
	\item 惟因函數之同數本從變數而生,故變數之同數與函數之長數比則爲\CJKmoveback{\equa{$\displaystyle\frac{\textit{辛}}{\textit{戌}'\MinusSign\textit{戌}}=\textit{二天}\PlusSign\textit{辛}$}}\CJKmove。\\
	設函數之式爲\CJKmoveback{\equa{$\textit{戌}=\textit{天}^\textit{三}$}}\CJKmove,令天長數爲辛,而以函數之新同數爲\CJKmoveback{\equa{$\textit{戌}'$}}\CJKmove 則\\
	\CJKmoveback{\equa{
			$\displaystyle\textit{戌}'=(\textit{天}\PlusSign\textit{辛})^\textit{三}\begin{cases}
	            =\textit{天}^\textit{三}\PlusSign\textit{三天}^\textit{二}\textit{辛}\PlusSign\textit{三天辛}^\textit{二}\PlusSign\textit{辛}^\textit{三}\\
	            =\textit{戌}\PlusSign\textit{三天}^\textit{二}\textit{辛}\PlusSign\textit{三天辛}^\textit{二}\PlusSign\textit{辛}^\textit{三}
	        \end{cases}$
	}}\CJKmove,而\CJKmoveback{\equa{$\displaystyle\frac{\textit{辛}}{\textit{戌}'\MinusSign\textit{戌}}=\textit{三天}^\textit{二}\PlusSign\textit{三天辛}\PlusSign\textit{辛}^\textit{二}$}}\CJKmove。可見天變爲\CJKmoveback{\equa{$\textit{天}\PlusSign\textit{辛}$}}\CJKmove 之時,其函數\CJKmoveback{\equa{$\textit{戌}$}}\CJKmove 必變爲\CJKmoveback{\equa{$\textit{戌}\PlusSign\textit{三天}^\textit{二}\textit{辛}\PlusSign\textit{三天辛}^\textit{二}\PlusSign\textit{辛}^\textit{三}$}}\CJKmove,其所長之數爲\CJKmoveback{\equa{$\textit{三天}^\textit{二}\textit{辛}\PlusSign\textit{三天辛}^\textit{二}\PlusSign\textit{辛}^\textit{三}$}}\CJKmove。此式中之各項皆爲辛之整方與他數相乘所成。又可見變數與函數之變比例,其式爲\CJKmoveback{\equa{$\textit{三天}^\textit{二}\textit{辛}\PlusSign\textit{三天辛}^\textit{二}\PlusSign\textit{辛}^\textit{三}$}}\CJKmove,其初項\CJKmoveback{\equa{$\textit{三天}^\textit{二}$}}\CJKmove 與天之長數辛無相關。\\
    設函數之式爲\CJKmoveback{\equa{$\textit{戌}=\textit{天}^\textit{四}$}}\CJKmove。令天之長數爲辛,而以\CJKmoveback{\equa{$\textit{戌}'$}}\CJKmove 爲函數之新同數,則\CJKmoveback{\equa{$\textit{戌}'=\textit{戌}\PlusSign\textit{四天}^\textit{三}\textit{辛}\PlusSign\textit{六天}^\textit{二}\textit{辛}^\textit{二}\PlusSign\textit{四天辛}^\textit{三}\PlusSign\textit{辛}^\textit{四}$}}\CJKmove,而\CJKmoveback{\equa{$\displaystyle\frac{\textit{辛}}{\textit{戌}'\MinusSign\textit{戌}}=\textit{四天}^\textit{三}\PlusSign\textit{六天}^\textit{二}\textit{辛}\PlusSign\textit{四天辛}^\textit{二}\PlusSign\textit{辛}^\textit{三}$}}\CJKmove。\\
    由此可見天若變爲\CJKmoveback{\equa{$\textit{天}\PlusSign\textit{辛}$}}\CJKmove,則其各函數之新同數如左:\\
    如\CJKmoveback{\equa{$\textit{戌}=\textit{天}^\textit{二}$}}\CJKmove,則\CJKmoveback{\equa{$\textit{戌}'=\textit{戌}\PlusSign\textit{二天辛}\PlusSign\textit{辛}^\textit{二}$}}\CJKmove。如\CJKmoveback{\equa{$\textit{戌}=\textit{天}^\textit{三}$}}\CJKmove,則\CJKmoveback{\equa{$\textit{戌}'=\textit{戌}\PlusSign\textit{三天}^\textit{二}\textit{辛}\PlusSign\textit{三天辛}^\textit{二}\PlusSign\textit{辛}^\textit{三}$}}\CJKmove。如\CJKmoveback{\equa{$\textit{戌}=\textit{天}^\textit{四}$}}\CJKmove,則\CJKmoveback{\equa{$\textit{戌}'=\textit{戌}\PlusSign\textit{四天}^\textit{三}\textit{辛}\PlusSign\textit{六天}^\textit{二}\textit{辛}^\textit{二}\PlusSign\textit{四天辛}^\textit{三}\PlusSign\textit{辛}^\textit{四}$}}\CJKmove。其餘類推。\\
    總言之,若以卯爲天之任何整指數,而令天之長數爲辛,又以巳午未申等字挨次而代辛之各方之倍數,則函數\CJKmoveback{\equa{$\textit{戌}=\textit{天}^\textit{卯}$}}\CJKmove 之新同數必爲\CJKmoveback{\equa{$\textit{戌}'=\textit{辛}\PlusSign\textit{巳辛}\PlusSign\textit{午辛}^\textit{二}\PlusSign\textit{未辛}^\textit{三}\PlusSign\textit{申辛}^\textit{四}\PlusSign\cdots\cdots$}}\CJKmove。由是知函數之新同數必爲級數,其初項戌爲函數之原同數,其餘各項爲天之長數辛之各整方,以巳午未申之類爲各倍數,其各倍數皆爲天之別種函數,其式亦從本函數而生。\\
    由以上各式又可見函數爲\CJKmoveback{\equa{$\textit{戌}=\textit{天}^\textit{二}$、$\textit{戌}=\textit{天}^\textit{三}$、$\textit{戌}=\textit{天}^\textit{四}$}}\CJKmove 之類,則其變數與函數之變比例必爲\CJKmoveback{\equa{$\displaystyle\frac{\textit{辛}}{\textit{戌}'\MinusSign\textit{戌}}=\textit{二天}\PlusSign\textit{辛}$、$\displaystyle\frac{\textit{辛}}{\textit{戌}'\MinusSign\textit{戌}}=\textit{三天}^\textit{二}\PlusSign\textit{三天辛}\PlusSign\textit{辛}^\textit{二}$、$\displaystyle\frac{\textit{辛}}{\textit{戌}'\MinusSign\textit{戌}}=\textit{四天}^\textit{三}\PlusSign\textit{六天}^\textit{二}\textit{辛}\PlusSign\textit{四天辛}^\textit{二}\PlusSign\textit{辛}^\textit{三}$}}\CJKmove 之類。總之若以卯爲天之整指數,則\CJKmoveback{\equa{$\textit{戌}=\textit{天}^\textit{卯}$}}\CJKmove 之變比例必爲\CJKmoveback{\equa{$\displaystyle\frac{\textit{辛}}{\textit{戌}'\MinusSign\textit{戌}}=\textit{巳}\PlusSign\textit{午辛}\PlusSign\textit{未辛}^\textit{二}\PlusSign\textit{申辛}^\textit{三}\PlusSign\cdots\cdots$}}\CJKmove。由此可見,變數天之長數與函數\CJKmoveback{\equa{$\textit{天}^\textit{卯}$}}\CJKmove 之長數其變比例\CJKmoveback{\equa{$\displaystyle\frac{\textit{辛}}{\textit{戌}'\MinusSign\textit{戌}}$}}\CJKmove 之同數\CJKmoveback{\equa{$\textit{巳}\PlusSign\textit{午辛}\PlusSign\textit{未辛}^\textit{二}\PlusSign\textit{申辛}^\textit{三}\PlusSign\cdots\cdots$}}\CJKmove 可分之爲兩式,其一式爲\CJKmoveback{\equa{$\textit{巳}$}}\CJKmove,此式與天之長數辛無相關;又一式爲\CJKmoveback{\equa{$\textit{午辛}\PlusSign\textit{未辛}^\textit{二}\PlusSign\textit{申辛}^\textit{三}\PlusSign\cdots\cdots$}}\CJKmove。卽\CJKmoveback{\equa{$\textit{辛}\left(\textit{午}\PlusSign\textit{未辛}\PlusSign\textit{申辛}^\textit{二}\PlusSign\cdots\cdots\right)$}}\CJKmove。此式因以辛爲乘數,故辛若變小,其數亦必隨辛而變小。如令辛爲任何小,則此式之數可小至甚近于〇。故此數可以不計,而以巳爲變比例之限。\\
    設有繁函數之式\CJKmoveback{\equa{$\textit{戌}=\textit{甲}\PlusSign\textit{乙天}\PlusSign\textit{丙天}^\textit{二}$}}\CJKmove,令天之長數爲辛,則天變爲\CJKmoveback{\equa{$\textit{天}\PlusSign\textit{辛}$}}\CJKmove 之時,其函數之同數必變爲\CJKmoveback{\equa{$\displaystyle\textit{戌}\begin{cases}
        =\textit{甲}\PlusSign\textit{乙}(\textit{天}\PlusSign\textit{辛})\PlusSign\textit{丙}(\textit{天}\PlusSign\textit{辛})^\textit{二}\\
        =\textit{甲}\PlusSign\textit{乙天}\PlusSign\textit{丙天}^\textit{二}\PlusSign(\textit{乙}\PlusSign\textit{二丙天})\textit{辛}\PlusSign\textit{丙辛}^\textit{二}
    \end{cases}$}}\CJKmove,故\\\CJKmoveback{\equa{$\textit{戌}'\MinusSign\textit{戌}=(\textit{乙}\PlusSign\textit{二丙天})\textit{辛}\PlusSign\textit{丙辛}^\textit{二}$}}\CJKmove,而\CJKmoveback{\equa{$\displaystyle\frac{\textit{辛}}{\textit{戌}'\MinusSign\textit{戌}}=\textit{乙}\PlusSign\textit{二丙天}\PlusSign\textit{丙辛}$}}\CJKmove。其\CJKmoveback{\equa{$\textit{乙}\PlusSign\textit{二丙天}$}}\CJKmove 爲本函數變比例之限。\\
    以此法徧試各種特設之函數,見其皆有相類之性情,所以例設如左。\\
    例曰:命任何自主之變數爲天,而令天之任何函數等于戌,則天變爲\CJKmoveback{\equa{$\textit{天}\PlusSign\textit{辛}$}}\CJKmove 之時,函數之新同數爲\CJKmoveback{\equa{$\textit{戌}'=\textit{戌}\PlusSign\textit{巳辛}\PlusSign\textit{午辛}^\textit{二}\PlusSign\textit{未辛}^\textit{三}\PlusSign\cdots\cdots$}}\CJKmove,其變數與函數之變比例爲\CJKmoveback{\equa{$\displaystyle\frac{\textit{辛}}{\textit{戌}'\MinusSign\textit{戌}}=\textit{巳}\PlusSign\textit{午辛}\PlusSign\textit{未辛}^\textit{二}\PlusSign\textit{申辛}^\textit{三}\PlusSign\cdots\cdots$}}\CJKmove。此式中之初項\CJKmoveback{\equa{$\textit{巳}$}}\CJKmove 爲變比例之限,無論何種函數,其限皆可依此比例求之。\\
    由以上所論變數之長數與函數之長數相關之理,可于算學中開出兩種極廣大極精微之法。\\
    其第一種爲有任何變數之任何函數而求其變數與函數變比例之限。\\
    其第二種爲有任何變數與函數變比例之限而求其函數之原式。\\
    此二種法,若細攷其根源,卽\uline{奈端}所謂正流數、反流數也;亦卽\uline{來本之}所謂微分算術、積分算術也;又卽\uline{拉果闌諸}所謂函數變例也。
\end{enumerate}

\section{論各種函數求微分之公法}

\begin{enumerate} [label={第\chinese*款}]
	\setcounter{enumi}{4}
	\item 若函數之式爲\CJKmoveback{\equa{$\textit{戌}=\textit{天}^\textit{二}$}}\CJKmove,令天變爲\CJKmoveback{\equa{$\textit{天}\PlusSign\textit{辛}$}}\CJKmove,則函數之新同數必爲\CJKmoveback{\equa{$\textit{戌}'=\textit{戌}\PlusSign\textit{二天辛}\PlusSign\textit{辛}^\textit{二}$}}\CJKmove,其與原同數之較爲\CJKmoveback{\equa{$\textit{戌}'\MinusSign\textit{戌}$}}\CJKmove,卽\CJKmoveback{\equa{$\textit{二天辛}\PlusSign\textit{辛}^\textit{二}$}}\CJKmove,此式之初項\CJKmoveback{\equa{$\textit{二天辛}$}}\CJKmove,名之曰溢率。\\
	依同理推之,若函數之式爲\CJKmoveback{\equa{$\textit{戌}=\textit{天}^\textit{三}$}}\CJKmove,令天變爲\CJKmoveback{\equa{$\textit{天}\PlusSign\textit{辛}$}}\CJKmove,則函數之新同數爲\CJKmoveback{\equa{$\textit{戌}'=\textit{戌}\PlusSign\textit{三天}^\textit{二}\textit{辛}\PlusSign\textit{三天辛}^\textit{二}\PlusSign\textit{辛}^\textit{三}$}}\CJKmove,其與原同數之較爲\CJKmoveback{\equa{$\textit{戌}'\MinusSign\textit{戌}$}}\CJKmove,卽\CJKmoveback{\equa{$\textit{三天}^\textit{二}\textit{辛}\PlusSign\textit{三天辛}^\textit{二}\PlusSign\textit{辛}^\textit{三}$}}\CJKmove。其初項\CJKmoveback{\equa{$\textit{三天}^\textit{二}\textit{辛}$}}\CJKmove 爲溢率。\\
	若函數之式爲\CJKmoveback{\equa{$\textit{戌}=\textit{天}^\textit{四}$}}\CJKmove,而天變爲\CJKmoveback{\equa{$\textit{天}\PlusSign\textit{辛}$}}\CJKmove,則函數之新同數與原同數之較爲\CJKmoveback{\equa{$\textit{四天}^\textit{三}\textit{辛}\PlusSign\textit{六天}^\textit{二}\textit{辛}^\textit{二}\PlusSign\textit{四天辛}^\textit{三}\PlusSign\textit{辛}^\textit{四}$}}\CJKmove,而其溢率爲\CJKmoveback{\equa{$\textit{四天}^\textit{三}\textit{辛}$}}\CJKmove。\\
	總言之,凡天之函數無論爲某方,恆可以\CJKmoveback{\equa{$\textit{天}\PlusSign\textit{辛}$}}\CJKmove 代其天,而變其函數之同數爲\CJKmoveback{\equa{$\textit{戌}\PlusSign\textit{巳辛}\PlusSign\textit{午辛}^\textit{二}\PlusSign\textit{未辛}^\textit{三}\PlusSign\cdots\cdots$}}\CJKmove,乃以原同數戌減之得\CJKmoveback{\equa{$\textit{巳辛}\PlusSign\textit{午辛}^\textit{二}\PlusSign\textit{未辛}^\textit{三}\PlusSign\cdots\cdots$}}\CJKmove,而取其初項\CJKmoveback{\equa{$\textit{巳辛}$}}\CJKmove 爲溢率。\\
	準此推之,則知天之溢率卽爲其長數辛。惟因函數之溢率每藉天之長數而生,故微分術中,恆以\CJKmoveback{\equa{$\textit{彳天}$}}\CJKmove 代天之溢率。其彳號者,非天之倍數,不過是天之溢率耳。溢率之名,本爲流數術中所用。而彳號者,卽微字之偏旁,故微分之術用之。\\
	如有式\CJKmoveback{\equa{$\textit{戌}=\textit{天}^\textit{二}$}}\CJKmove,則\CJKmoveback{\equa{$\textit{彳戌}=\textit{二天彳天}$}}\CJKmove。此式之意謂戌之微分等于\CJKmoveback{\equa{$\textit{二天}$}}\CJKmove 乘天之微分。猶言函數戌之溢率等于以\CJKmoveback{\equa{$\textit{二天}$}}\CJKmove 乘其天之溢率也。如有式\CJKmoveback{\equa{$\textit{戌}=\textit{天}^\textit{三}$}}\CJKmove,則\CJKmoveback{\equa{$\textit{彳戌}=\textit{三天}^\textit{二}\textit{彳天}$}}\CJKmove。此式之意謂戌之微分等于\CJKmoveback{\equa{$\textit{三}\textit{天}^\textit{二}$}}\CJKmove 乘天之微分。猶言函數戌之溢率等于以\CJKmoveback{\equa{$\textit{三}\\\textit{天}^\textit{二}$}}\CJKmove 乘其天之溢率也。
	\item 惟因每遇\CJKmoveback{\equa{$\textit{戌}=\textit{天}^\textit{二}$}}\CJKmove,則\CJKmoveback{\equa{$\textit{彳戌}=\textit{二天彳天}$}}\CJKmove,所以可寫之如\CJKmoveback{\equa{$\displaystyle\frac{\textit{彳天}}{\textit{彳戌}}=\textit{二天}$}}\CJKmove。此爲天微分之倍數,亦謂之微係數。\\
	又依前法推之,如函數之式爲\CJKmoveback{\equa{$\textit{戌}=\textit{天}^\textit{三}$}}\CJKmove,則\CJKmoveback{\equa{$\textit{彳戌}=\textit{三天}^\textit{二}\textit{彳天}$}}\CJKmove,而\CJKmoveback{\equa{$\displaystyle\frac{\textit{彳天}}{\textit{彳戌}}=\textit{三天}^\textit{二}$}}\CJKmove。其\CJKmoveback{\equa{$\textit{三天}^\textit{二}$}}\CJKmove 爲原函數\CJKmoveback{\equa{$\textit{戌}=\textit{天}^\textit{三}$}}\CJKmove 之微係數。\\
	總言之,無論何種函數之微係數,皆可以\CJKmoveback{\equa{$\displaystyle\frac{\textit{彳天}}{\textit{彳戌}}$}}\CJKmove 代之。而函數之新同數爲\CJKmoveback{\equa{$\textit{戌}\PlusSign\textit{巳辛}\PlusSign\textit{午辛}^\textit{二}\PlusSign\cdots\cdots$}}\CJKmove,所以\CJKmoveback{\equa{$\displaystyle\frac{\textit{彳天}}{\textit{彳戌}}=\textit{巳}$}}\CJKmove,其巳爲天之他函數,其形每隨函數之式而變。如之戌之同數爲何式,則其巳之同數卽易求得。\\
	凡函數之欲求微分者,先于其式之左旁作一彳號以記之。如有\CJKmoveback{\equa{$[(\textit{甲}\PlusSign\textit{天})(\textit{乙}^\textit{二}\MinusSign\textit{天}^\textit{二})]$}}\CJKmove 式,欲求其微分,則可先作\CJKmoveback{\equa{$\textit{彳}[(\textit{甲}\PlusSign\textit{天})(\textit{乙}^\textit{二}\MinusSign\textit{天}^\textit{二})]$}}\CJKmove。\\
	凡函數之欲求微係數者,于其式之左旁作彳號,又以\CJKmoveback{\equa{$\textit{彳天}$}}\CJKmove 爲其分母。如有\CJKmoveback{\equa{$[(\textit{甲}\PlusSign\textit{天})\\(\textit{乙}^\textit{二}\MinusSign\textit{天}^\textit{二})]$}}\CJKmove 式,欲求其微係數,則作\CJKmoveback{\equa{$\displaystyle\frac{\textit{彳天}}{\textit{彳}[(\textit{甲}\PlusSign\textit{天})(\textit{乙}^\textit{二}\MinusSign\textit{天}^\textit{二})]}$}}\CJKmove。\\
	從以上各款諸說,易知求微分之公法。\\
	法曰:無論天之任何函數,欲求微分,則以\CJKmoveback{\equa{$\textit{天}\PlusSign\textit{辛}$}}\CJKmove 代其原式中之天而詳之。依辛之整方數自小而大序之,取其初有辛之項,而以\CJKmoveback{\equa{$\textit{彳天}$}}\CJKmove 代其辛即得。\\
	如有式\CJKmoveback{\equa{$\textit{戌}=\textit{甲天}\PlusSign\textit{乙天}^\textit{二}$}}\CJKmove,欲求其微係數,則以\CJKmoveback{\equa{$\textit{天}\PlusSign\textit{辛}$}}\CJKmove 代其天,而令函數之新同數爲\CJKmoveback{\equa{$\textit{戌}'$}}\CJKmove,則得\CJKmoveback{\equa{$\displaystyle\textit{戌}'\begin{cases}
		=\textit{甲}(\textit{天}\PlusSign\textit{辛})\PlusSign\textit{乙}(\textit{天}\PlusSign\textit{辛})^\textit{二}\\
		=\textit{甲}\textit{天}\PlusSign\textit{乙天}^\textit{二}\PlusSign(\textit{甲}\PlusSign\textit{二乙天})\textit{辛}\PlusSign\textit{乙辛}^\textit{二}\\
		=\textit{戌}\PlusSign(\textit{甲}\PlusSign\textit{二乙天})\textit{辛}\PlusSign\textit{乙辛}^\textit{二}
	\end{cases}$}}\CJKmove。取其初有辛之項\CJKmoveback{\equa{$(\textit{甲}\PlusSign\textit{二乙天})\textit{辛}$}}\CJKmove,以\CJKmoveback{\equa{$\textit{彳天}$}}\CJKmove 代其辛,卽得\CJKmoveback{\equa{$\textit{彳戌}=(\textit{甲}\PlusSign\textit{二乙天})\textit{彳天}$}}\CJKmove,故其\CJKmoveback{\equa{$\displaystyle\frac{\textit{彳天}}{\textit{彳戌}}=\textit{甲}\PlusSign\textit{二乙天}$}}\CJKmove。
    \item 上款之法,必令天變爲\CJKmoveback{\equa{$\textit{天}\PlusSign\textit{辛}$}}\CJKmove,而詳其函數之同數爲級數,此乃論其立法之理當如是也。惟求得級數之後,所用者僅爲其辛一方之項,則但能求得此項巳足用矣。前于第四款中言此項之倍數爲變數與函數變比例之限,又于第六款中言此項之倍數謂之微係數。然則求函數之微係數與求變數與函數變比例之限,其法本無異也。\\
    如有式\CJKmoveback{\equa{$\displaystyle\textit{戌}=\frac{\textit{天}}{\textit{甲}^\textit{二}}$}}\CJKmove,則\CJKmoveback{\equa{$\displaystyle\textit{戌}'=\frac{\textit{天}\PlusSign\textit{辛}}{\textit{甲}^\textit{二}}$}}\CJKmove。而\CJKmoveback{\equa{$\displaystyle\textit{戌}\MinusSign\textit{戌}'=\frac{\textit{天}}{\textit{甲}^\textit{二}}\MinusSign\frac{\textit{天}\PlusSign\textit{辛}}{\textit{甲}^\textit{二}}$}}\CJKmove,卽\CJKmoveback{\equa{$\displaystyle\textit{戌}\MinusSign\textit{戌}'=\frac{\textit{天}(\textit{天}\PlusSign\textit{辛})}{\MinusSign\textit{甲}^\textit{二}\textit{辛}}$}}\CJKmove,故其變比例之式爲\CJKmoveback{\equa{$\displaystyle\frac{\textit{辛}}{\textit{戌}\MinusSign\textit{戌}'}=\frac{\textit{天}(\textit{天}\PlusSign\textit{辛})}{\MinusSign\textit{甲}^\textit{二}}$}}\CJKmove。惟此式可不待詳爲級數而始得其變比例之限,因可見辛愈小,則其式愈近于\CJKmoveback{\equa{$\displaystyle\MinusSign\frac{\textit{天}^\textit{二}}{\textit{甲}^\textit{二}}$}}\CJKmove,故此式必卽爲\CJKmoveback{\equa{$\displaystyle\frac{\textit{彳天}}{\textit{彳戌}}$}}\CJKmove 之同數,所以得\CJKmoveback{\equa{$\displaystyle\textit{彳戌}=\MinusSign\frac{\textit{天}^\textit{二}}{\textit{甲}^\textit{二}\textit{彳天}}$}}\CJKmove。\\
    若以戌爲任何函數之原同數,而以\CJKmoveback{\equa{$\textit{天}\PlusSign\textit{辛}$}}\CJKmove 代其天,則其函數之新同數爲\CJKmoveback{\equa{$\textit{戌}'=\textit{戌}\PlusSign\textit{巳辛}\PlusSign\textit{午辛}^\textit{二}\PlusSign\textit{未辛}^\textit{三}\PlusSign\cdots\cdots$}}\CJKmove,卽得\CJKmoveback{\equa{$\displaystyle\frac{\textit{辛}}{\textit{戌}'\MinusSign\textit{戌}}=\textit{巳}\PlusSign\textit{午辛}\PlusSign\textit{未辛}^\textit{二}\PlusSign\cdots\cdots$}}\CJKmove。惟因\CJKmoveback{\equa{$\displaystyle\frac{\textit{辛}}{\textit{戌}'\MinusSign\textit{戌}}$}}\CJKmove 之限爲\CJKmoveback{\equa{$\textit{巳}$}}\CJKmove,故得\CJKmoveback{\equa{$\displaystyle\frac{\textit{彳天}}{\textit{彳戌}}=\textit{巳}$}}\CJKmove,而\CJKmoveback{\equa{$\displaystyle\textit{彳戌}=\textit{巳彳天}$}}\CJKmove。\\
    由此得一解曰:凡微分之術,其意專爲求任何變數與函數同時變大之限耳。\\
    凡求任何函數之微分,不過將一尋常之代數式另用他法以變化之。因其變化之法與代數中常用之法異,則不得不另有一名以別之,故謂之微分術。
    \item 凡變數與函數變比例之限,無論以何數爲主,其形必同。\\
    如戌爲天之函數,若天變爲\CJKmoveback{\equa{$\textit{天}\PlusSign\textit{辛}$}}\CJKmove,則戌變爲\CJKmoveback{\equa{$\textit{戌}\PlusSign\textit{巳辛}\PlusSign\textit{午辛}^\textit{二}\PlusSign\textit{未辛}^\textit{三}\PlusSign\cdots\cdots$}}\CJKmove。其巳、午、未各數俱爲天之他函數,從本函數所生。如令其\CJKmoveback{\equa{$\textit{巳辛}\PlusSign\textit{午辛}^\textit{二}\PlusSign\textit{未辛}^\textit{三}\PlusSign\cdots\cdots=\textit{子}$}}\CJKmove,則天與戌同時變大之數爲辛與子。如依\uwave{代數術}第一百六十三款之法反求其級數,則得\CJKmoveback{\equa{$\displaystyle\textit{辛}=\frac{\textit{巳}}{\textit{一}}\textit{子}\MinusSign\frac{\textit{巳}^\textit{二}}{\textit{午}}\textit{子}^\textit{二}\PlusSign\cdots\cdots$}}\CJKmove,故其變數與函數之公比例爲\CJKmoveback{\equa{$\displaystyle\frac{\textit{子}}{\textit{辛}}=\frac{\textit{巳}}{\textit{一}}\MinusSign\frac{\textit{巳}^\textit{二}}{\textit{午}}\textit{子}\PlusSign\cdots\cdots$}}\CJKmove。故\CJKmoveback{\equa{$\displaystyle\frac{\textit{子}}{\textit{辛}}$}}\CJKmove 之限爲\CJKmoveback{\equa{$\displaystyle\frac{\textit{巳}}{\textit{一}}$}}\CJKmove,而\CJKmoveback{\equa{$\displaystyle\frac{\textit{辛}}{\textit{子}}$}}\CJKmove 之限爲\CJKmoveback{\equa{$\textit{巳}$}}\CJKmove,此卽\CJKmoveback{\equa{$\displaystyle\frac{\textit{一}}{\textit{巳}}$}}\CJKmove 也。
    \item 由此易知,凡有相等之函數,則其微係數亦必相等。\\
    如戌與亥爲兩函數,而\CJKmoveback{\equa{$\textit{戌}=\textit{亥}$}}\CJKmove,其天變爲\CJKmoveback{\equa{$\textit{天}\PlusSign\textit{辛}$}}\CJKmove 之時,戌變爲\CJKmoveback{\equa{$\textit{戌}'$}}\CJKmove,亥變爲\CJKmoveback{\equa{$\textit{亥}'$}}\CJKmove,則\CJKmoveback{\equa{$\textit{戌}'=\textit{亥}'$}}\CJKmove,而\CJKmoveback{\equa{$\textit{戌}'\MinusSign\textit{戌}=\textit{亥}'\MinusSign\textit{亥}$}}\CJKmove,故\CJKmoveback{\equa{$\displaystyle\frac{\textit{辛}}{\textit{戌}'\MinusSign\textit{戌}}=\frac{\textit{辛}}{\textit{亥}'\MinusSign\textit{亥}}$}}\CJKmove。如以巳與午各爲其變比例之限,則\CJKmoveback{\equa{$\textit{巳}=\textit{午}$}}\CJKmove。故\CJKmoveback{\equa{$\textit{巳彳天}=\textit{午彳天}$}}\CJKmove,而\CJKmoveback{\equa{$\textit{彳戌}=\textit{彳亥}$}}\CJKmove。\\
    由此可見,凡函數之式無論如何改形,若其同數無異者,則其微係數必同。\\
    如函數之原式爲\CJKmoveback{\equa{$\textit{天}^\textit{三}\PlusSign\textit{甲}^\textit{三}$}}\CJKmove,若改其形爲\CJKmoveback{\equa{$(\textit{天}\PlusSign\textit{甲})(\textit{天}^\textit{二}\MinusSign\textit{甲天}\PlusSign\textit{甲}^\textit{二})$}}\CJKmove,則此式之微係數與原式之微係數必無異。\\
    惟此理,若反言之而謂相等之微係數必從相等之函數而生,則不盡然。\\
    如函數之式爲\CJKmoveback{\equa{$\textit{戌}=\textit{甲}\PlusSign\textit{乙天}$}}\CJKmove,令天變爲\CJKmoveback{\equa{$\textit{天}\PlusSign\textit{辛}$}}\CJKmove,而戌變爲\CJKmoveback{\equa{$\textit{戌}'$}}\CJKmove,則\CJKmoveback{\equa{$\textit{戌}'=\textit{甲}\PlusSign\textit{乙天}\PlusSign\textit{乙辛}=\textit{戌}\PlusSign\textit{乙辛}$}}\CJKmove,故得\CJKmoveback{\equa{$\displaystyle\frac{\textit{辛}}{\textit{戌}'\MinusSign\textit{戌}}=\textit{乙}$}}\CJKmove。觀此可知,其常數之項甲不能入變比例之限內,故其微係數必與\CJKmoveback{\equa{$\textit{戌}=\textit{乙天}$}}\CJKmove 之微係數無異。惟微係數乙既能屬於本函數\CJKmoveback{\equa{$\textit{甲}\PlusSign\textit{乙天}$}}\CJKmove,又能屬於他函數\CJKmoveback{\equa{$\textit{乙天}$}}\CJKmove,所以有下例。\\
    例曰:凡變數與常數相加減之函數,其微係數中不見其加減之常數。惟變數與常數相乘除者,則其微係數中有常數爲倍數。
\end{enumerate}

\section{求兩函數相乘積之微分}

\begin{enumerate} [label={第\chinese*款}]
	\setcounter{enumi}{9}
	\item 凡變數之函數,無論其形如何,皆可以第六款之公法求其微分。然不如每種異形之函數各設一專法以求之,則更簡捷。\\
	如有式\CJKmoveback{\equa{$\textit{戌}=\textit{未申}$}}\CJKmove,其未與申各爲天之函數。今欲得一法專能求未、申相乘積之微分,若令天變爲\CJKmoveback{\equa{$\textit{天}\PlusSign\textit{辛}$}}\CJKmove,則未、申二函數必變爲\CJKmoveback{\equa{$\textit{未}'=\textit{未}\PlusSign\textit{巳辛}\PlusSign\textit{午辛}^\textit{二}\PlusSign\cdots\cdots$、$\textit{申}'=\textit{申}\PlusSign\textit{巳}'\textit{辛}\PlusSign\textit{午}'\textit{辛}^\textit{二}\PlusSign\cdots\cdots$}}\CJKmove,其巳、午爲由未函數所得之天之他函數;其\CJKmoveback{\equa{$\textit{巳}'\textit{、}\textit{午}'$}}\CJKmove 爲由申函數所得之天之他函數。\\
	再令\CJKmoveback{\equa{$\textit{戌}'=\textit{未}'\textit{申}'$}}\CJKmove,則依法得\CJKmoveback{\equa{$\textit{戌}'=\textit{未}'\textit{申}'=\textit{未申}\PlusSign(\textit{未巳}'\PlusSign\textit{申巳})\textit{辛}\PlusSign(\textit{未午}'\PlusSign\textit{巳巳}'\PlusSign\textit{申午})\textit{辛}^\textit{二}\PlusSign\cdots\cdots$}}\CJKmove。以戌代其\CJKmoveback{\equa{$\textit{未申}$}}\CJKmove,移項而以辛約之,則得\CJKmoveback{\equa{$\displaystyle\frac{\textit{辛}}{\textit{戌}'\MinusSign\textit{戌}}=\textit{未巳}'\PlusSign\textit{申巳}\PlusSign(\textit{未午}'\PlusSign\textit{巳巳}'\PlusSign\textit{申午})\textit{辛}\PlusSign\cdots\cdots$}}\CJKmove。此式中之\CJKmoveback{\equa{$\textit{未巳}'$}}\CJKmove 與\CJKmoveback{\equa{$\textit{申巳}$}}\CJKmove 兩項乃天之他函數,而與辛無涉者,其餘各項俱有辛之各方爲乘數。設辛爲甚微,則其所乘之衆項亦甚微,故其變比例之限爲\CJKmoveback{\equa{$\displaystyle\frac{\textit{辛}}{\textit{戌}'\MinusSign\textit{戌}}=\textit{未巳}'\PlusSign\textit{申巳}$}}\CJKmove。惟因\CJKmoveback{\equa{$\displaystyle\textit{巳}=\frac{\textit{辛}}{\textit{未}'\MinusSign\textit{未}}$、$\displaystyle\textit{巳}'=\frac{\textit{辛}}{\textit{申}'\MinusSign\textit{申}}$}}\CJKmove,故可依第六款之法以\CJKmoveback{\equa{$\displaystyle\frac{\textit{彳天}}{\textit{彳戌}}$}}\CJKmove 代其\CJKmoveback{\equa{$\displaystyle\frac{\textit{辛}}{\textit{戌}'\MinusSign\textit{戌}}$}}\CJKmove;以\CJKmoveback{\equa{$\displaystyle\frac{\textit{彳天}}{\textit{彳未}}$}}\CJKmove 代其\CJKmoveback{\equa{$\displaystyle\frac{\textit{辛}}{\textit{未}'\MinusSign\textit{未}}$}}\CJKmove;以\CJKmoveback{\equa{$\displaystyle\frac{\textit{彳天}}{\textit{彳申}}$}}\CJKmove 代其\CJKmoveback{\equa{$\displaystyle\frac{\textit{辛}}{\textit{申}'\MinusSign\textit{申}}$}}\CJKmove。則得\CJKmoveback{\equa{$\displaystyle\frac{\textit{彳天}}{\textit{彳戌}}=\textit{未}\frac{\textit{彳天}}{\textit{彳申}}\PlusSign\textit{申}\frac{\textit{彳天}}{\textit{彳未}}$}}\CJKmove,而\CJKmoveback{\equa{$\textit{彳戌}=\textit{未彳申}\PlusSign\textit{申彳未}$}}\CJKmove。故得專法如左:\\
	凡求同變彼此兩函數相乘積之微分,法將此函數乘彼函數之微分,又將彼函數乘此函數之微分,而以乘得之兩式相加卽得。
\end{enumerate}

\section{求多函數連乘積之微分}

\begin{enumerate} [label={第\chinese*款}]
	\setcounter{enumi}{10}
	\item 前款論函數之式爲\CJKmoveback{\equa{$\textit{戌}=\textit{未申}$}}\CJKmove,則\CJKmoveback{\equa{$\textit{彳戌}=\textit{未彳申}\PlusSign\textit{申彳未}$}}\CJKmove,所以\CJKmoveback{\equa{$\displaystyle\frac{\textit{戌}}{\textit{彳戌}}=\frac{\textit{未}}{\textit{彳未}}\PlusSign\frac{\textit{申}}{\textit{彳申}}$}}\CJKmove。由此推之,如戌爲三箇同變數之函數連乘如\CJKmoveback{\equa{$\textit{未酉亥}$}}\CJKmove,可令其\CJKmoveback{\equa{$\textit{申}=\textit{酉亥}$}}\CJKmove,則亦能爲\CJKmoveback{\equa{$\textit{戌}=\textit{未申}$}}\CJKmove 而\CJKmoveback{\equa{$\displaystyle\frac{\textit{戌}}{\textit{彳戌}}=\frac{\textit{未}}{\textit{彳未}}\PlusSign\frac{\textit{申}}{\textit{彳申}}$}}\CJKmove。惟因\CJKmoveback{\equa{$\textit{申}=\textit{酉亥}$}}\CJKmove,可依同例得\CJKmoveback{\equa{$\displaystyle\frac{\textit{申}}{\textit{彳申}}=\frac{\textit{酉}}{\textit{彳酉}}\PlusSign\frac{\textit{亥}}{\textit{彳亥}}$}}\CJKmove,所以\CJKmoveback{\equa{$\displaystyle\frac{\textit{戌}}{\textit{彳戌}}=\frac{\textit{未}}{\textit{彳未}}\PlusSign\frac{\textit{酉}}{\textit{彳酉}}\PlusSign\frac{\textit{亥}}{\textit{彳亥}}$}}\CJKmove。若仍將\CJKmoveback{\equa{$\textit{未申}$}}\CJKmove 代還其戌而以常法化之,則爲\CJKmoveback{\equa{$\textit{彳戌}=\textit{酉亥彳未}\PlusSign\textit{未亥彳酉}\PlusSign\textit{未酉彳亥}$}}\CJKmove。故得專法如左:\\
	凡求同變數之若干函數連乘積之微分,法以各本函數之微分與其餘之各他函數連乘,而以各乘得之式相加卽得。\\
	此法易用以總式以明之。無論其同變數之函數有若干數連乘,皆可以一例推之:\\
	如多函數連乘之式爲\CJKmoveback{\equa{$\textit{未申酉亥}$}}\CJKmove,則其微分之式爲\CJKmoveback{\equa{$\displaystyle\textit{彳}(\textit{未申酉亥})=\textit{未申酉亥}\left[\frac{\textit{未}}{\textit{彳未}}\PlusSign\right.$\\$\displaystyle\left.\frac{\textit{申}}{\textit{彳申}}\PlusSign\frac{\textit{酉}}{\textit{彳酉}}\PlusSign\frac{\textit{亥}}{\textit{彳亥}}\right]$}}\CJKmove。
\end{enumerate}

\section{求變數之分函數微分}

\begin{enumerate} [label={第\chinese*款}]
	\setcounter{enumi}{11}
	\item 若有分數之式,其母子爲同變數之各函數,則欲得其求微分之專法,可令\CJKmoveback{\equa{$\displaystyle\textit{戌}=\frac{\textit{申}}{\textit{未}}$}}\CJKmove,則\CJKmoveback{\equa{$\textit{未}=\textit{戌申}$}}\CJKmove,而\CJKmoveback{\equa{$\textit{彳未}=\textit{戌彳申}\PlusSign\textit{申彳戌}$}}\CJKmove。乃以其戌之同數\CJKmoveback{\equa{$\displaystyle\frac{\textit{申}}{\textit{未}}$}}\CJKmove 代其戌,則\CJKmoveback{\equa{$\displaystyle\textit{彳未}=\frac{\textit{申}}{\textit{未彳申}}\PlusSign\textit{申彳戌}$}}\CJKmove,而\CJKmoveback{\equa{$\displaystyle\textit{彳戌}=\frac{\textit{申}^\textit{二}}{\textit{申彳未}\MinusSign\textit{未彳申}}$}}\CJKmove。故得專法如左:\\
	凡同變量之函數,若爲分數,則求微分之法可將分母乘其分子之微分,乃以分子乘分母之微分減之,而以分母之平方約之。\\
	此法亦可用一總式以明之。\\
	惟因\CJKmoveback{\equa{$\displaystyle\textit{戌}=\frac{\textit{申}}{\textit{未}}$}}\CJKmove 之微分爲\CJKmoveback{\equa{$\displaystyle\frac{\textit{戌}}{\textit{彳戌}}=\frac{\textit{未}}{\textit{彳未}}\MinusSign\frac{\textit{申}}{\textit{彳申}}$}}\CJKmove,所以\CJKmoveback{\equa{$\displaystyle\textit{戌}=\frac{\textit{亥地}}{\textit{未申酉}}$}}\CJKmove 之微分爲\CJKmoveback{\equa{$\displaystyle\frac{\textit{戌}}{\textit{彳戌}}=\frac{\textit{未}}{\textit{彳未}}\PlusSign\frac{\textit{申}}{\textit{彳申}}\PlusSign\frac{\textit{酉}}{\textit{彳酉}}\MinusSign\frac{\textit{亥}}{\textit{彳亥}}\MinusSign\frac{\textit{地}}{\textit{彳地}}$}}\CJKmove。由此推之,則\CJKmoveback{\equa{$\displaystyle\textit{戌}=\frac{\textit{亥地}}{\textit{未申酉}}$}}\CJKmove 之微分當爲\CJKmoveback{\equa{$\displaystyle\frac{\textit{戌}}{\textit{彳戌}}=\frac{\textit{未申酉}}{\textit{彳}(\textit{未申酉})}\MinusSign\frac{\textit{亥地}}{\textit{彳}(\textit{亥地})}$}}\CJKmove。又依第十一款之例,其\CJKmoveback{\equa{$\displaystyle\frac{\textit{未申酉}}{\textit{彳}(\textit{未申酉})}=\frac{\textit{未}}{\textit{彳未}}\PlusSign\frac{\textit{申}}{\textit{彳申}}\PlusSign\frac{\textit{酉}}{\textit{彳酉}}$}}\CJKmove,而\CJKmoveback{\equa{$\displaystyle\frac{\textit{亥地}}{\textit{彳}(\textit{亥地})}=\frac{\textit{亥}}{\textit{彳亥}}\PlusSign\frac{\textit{地}}{\textit{彳地}}$}}\CJKmove,故得\CJKmoveback{\equa{$\displaystyle\frac{\textit{戌}}{\textit{彳戌}}=\frac{\textit{未}}{\textit{彳未}}\PlusSign\frac{\textit{申}}{\textit{彳申}}\PlusSign\frac{\textit{酉}}{\textit{彳酉}}\MinusSign\frac{\textit{亥}}{\textit{彳亥}}\MinusSign\frac{\textit{地}}{\textit{彳地}}$}}\CJKmove。由是知若有分數之函數,其分母分子爲任若干同變數之函數連乘之積如\CJKmoveback{\equa{$\displaystyle\frac{\textit{亥地}}{\textit{未申酉}}$}}\CJKmove 者,可以\CJKmoveback{\equa{$\displaystyle\textit{彳}\left(\frac{\textit{亥地}}{\textit{未申酉}}\right)=\frac{\textit{亥地}}{\textit{未申酉}}\left[\frac{\textit{未}}{\textit{彳未}}\PlusSign\frac{\textit{申}}{\textit{彳申}}\PlusSign\frac{\textit{酉}}{\textit{彳酉}}\MinusSign\frac{\textit{亥}}{\textit{彳亥}}\MinusSign\frac{\textit{地}}{\textit{彳地}}\right]$}}\CJKmove 之式明之。
	\item 茲欲攷任何函數地之任何乘方求微分之專法。\textsuperscript{其地或爲自變之數,或爲他變之函數,皆可。}\\
	先設\CJKmoveback{\equa{$\displaystyle\textit{戌}=\textit{地}^\textit{卯}$}}\CJKmove,其卯爲任何整數,則其函數之詳式必有卯箇地連乘如\CJKmoveback{\equa{$\displaystyle\textit{戌}=\textit{地}\cdot\textit{地}\cdot\textit{地}\cdot\textit{地}\cdots\cdots$}}\CJKmove。則依第十一款之例,\CJKmoveback{\equa{$\displaystyle\frac{\textit{戌}}{\textit{彳戌}}=\frac{\textit{地}}{\textit{彳地}}\PlusSign\frac{\textit{地}}{\textit{彳地}}\PlusSign\frac{\textit{地}}{\textit{彳地}}\PlusSign\cdots\cdots$}}\CJKmove,其項必有卯數。故\CJKmoveback{\equa{$\displaystyle\frac{\textit{戌}}{\textit{彳戌}}=\frac{\textit{地}}{\textit{卯彳地}}$}}\CJKmove,而得\CJKmoveback{\equa{$\displaystyle\textit{彳戌}=\frac{\textit{地}}{\textit{卯彳地}}\textit{戌}=\textit{卯地}^{\textit{卯}\MinusSign\textit{一}}\textit{彳地}$}}\CJKmove。\\
	設函數爲分指數如\CJKmoveback{\equa{$\displaystyle\textit{戌}=\textit{地}^\frac{\textit{卯}}{\textit{寅}}$}}\CJKmove,則\CJKmoveback{\equa{$\displaystyle\textit{戌}^\textit{卯}=\textit{地}^\textit{寅}$}}\CJKmove。若依第九款之例,則得\CJKmoveback{\equa{$\displaystyle\textit{卯戌}^{\textit{卯}\MinusSign\textit{一}}\textit{彳戌}=\textit{寅地}^{\textit{寅}\MinusSign\textit{一}}\textit{彳地}$}}\CJKmove,而\CJKmoveback{\equa{$\displaystyle\textit{彳戌}=\frac{\textit{卯}}{\textit{寅}}\times\frac{\textit{戌}^{\textit{卯}\MinusSign\textit{一}}}{\textit{地}^{\textit{寅}\MinusSign\textit{一}}}\textit{彳地}$}}\CJKmove。惟因\CJKmoveback{\equa{$\displaystyle\textit{戌}=\textit{地}^\frac{\textit{卯}}{\textit{寅}}$、$\displaystyle\textit{戌}^{\textit{卯}\MinusSign\textit{一}}=\textit{地}^{\textit{寅}\MinusSign\frac{\textit{卯}}{\textit{寅}}}$}}\CJKmove,所以\CJKmoveback{\equa{$\displaystyle\frac{\textit{戌}^{\textit{卯}\MinusSign\textit{一}}}{\textit{地}^{\textit{寅}\MinusSign\textit{一}}}=\textit{地}^{\frac{\textit{卯}}{\textit{寅}}\MinusSign\textit{一}}$}}\CJKmove,而\CJKmoveback{\equa{$\displaystyle\textit{彳戌}=\frac{\textit{卯}}{\textit{寅}}\textit{地}^{\frac{\textit{卯}}{\textit{寅}}\MinusSign\textit{一}}\textit{彳地}$}}\CJKmove。\\
	再設卯爲負指數,無論爲整數爲分數,則\CJKmoveback{\equa{$\displaystyle\textit{戌}=\textit{地}^{\MinusSign\textit{卯}}$}}\CJKmove,卽\CJKmoveback{\equa{$\displaystyle\textit{戌}=\frac{\textit{地}^\textit{卯}}{\textit{一}}$}}\CJKmove。若依第十二款之例,因其分子爲常數,故其分子之微分當爲〇,而得\CJKmoveback{\equa{$\displaystyle\textit{彳戌}=\MinusSign\frac{\textit{地}^{\textit{二卯}}}{\textit{卯地}^{\textit{卯}\MinusSign\textit{一}}\textit{彳地}}$}}\CJKmove,卽\CJKmoveback{\equa{$\displaystyle\textit{彳戌}=\MinusSign\textit{卯地}^{\MinusSign\textit{卯}\MinusSign\textit{一}}\textit{彳地}$}}\CJKmove。\\
	合觀本款之各式,可見地之指數卯無論爲正爲負爲整爲分,其微分之式必爲\CJKmoveback{\equa{$\displaystyle\textit{彳}\left(\textit{地}^\textit{卯}\right)=\textit{卯地}^{\textit{卯}\MinusSign\textit{一}}\textit{彳地}$}}\CJKmove。故得專法如左:\\
	凡求函數乘方之微分,法將其原指數以一減之爲新指數,而以原指數爲其倍數,又以變數之微分乘之卽得。惟其原函數若本有常數爲倍數者,則其原倍數必仍在乘數之中。\\
	如函數之式爲\CJKmoveback{\equa{$\displaystyle\textit{戌}=\textit{甲天}^\textit{卯}$}}\CJKmove,則其微分之式爲\CJKmoveback{\equa{$\displaystyle\textit{彳}\left(\textit{甲天}^\textit{卯}\right)=\textit{卯甲天}^{\textit{卯}\MinusSign\textit{一}}\textit{彳天}$}}\CJKmove。\\
	求函數諸乘方之微分更有簡便之法,可藉\uwave{代數術}第一百六十款與一百六十一款之二項例而得。所以于此不論者,因二項之例亦可由微分而得,余欲用微分之術證明二項之例以便于用,故俟後詳論之。
\end{enumerate}

\section{求變數平方根之微分}

\setlength{\leftskip}{25pt}

\noindent
用前法已能求各負方之微分,惟因\CJKmoveback{\equa{$\sqrt{\textit{地}}=\textit{地}^{\frac{\textit{二}}{\textit{一}}}$}}\CJKmove 爲微分術中常見之式,所以必更設一最易之專法以便于用。\\
依本款求諸乘方微分之法,\CJKmoveback{\equa{$\displaystyle\textit{彳}\left(\textit{地}^{\frac{\textit{二}}{\textit{一}}}\right)=\frac{\textit{二}}{\textit{一}}\textit{地}^{\frac{\textit{二}}{\textit{一}}\MinusSign\textit{一}}\textit{彳地}$}}\CJKmove,卽\CJKmoveback{\equa{$\displaystyle\textit{彳}\left(\textit{地}^{\frac{\textit{二}}{\textit{一}}}\right)=\frac{\textit{二}}{\textit{一}}\textit{地}^{\MinusSign\frac{\textit{二}}{\textit{一}}}\textit{彳地}$}}\CJKmove,卽\CJKmoveback{\equa{$\displaystyle\textit{彳}\left(\sqrt{\textit{地}}\right)=\frac{\textit{二}\sqrt{\textit{地}}}{\textit{彳地}}$}}\CJKmove。所以得專法如左:\\
凡求函數平方根之微分,法以函數之微分爲實而二倍其原函數之平方根以約之卽得。

\setlength{\leftskip}{0pt}

\section{求重函數之微分}

\begin{enumerate} [label={第\chinese*款}]
	\setcounter{enumi}{13}
	\item 設有地爲天之函數
\end{enumerate}