%!TEX program = xelatex
%!TEX encoding = UTF-8

\documentclass[UTF8, nofont, landscape, a5paper]{ctexbook}

\usepackage{ctex}
\usepackage{xeCJK}
\usepackage{amsmath}
\usepackage{amsfonts}
\usepackage{amssymb}
\usepackage{graphicx,stackengine,scalerel}
\def\tang{\ThisStyle{\abovebaseline[0pt]{\scalebox{-1}{$\SavedStyle\perp$}}}}
\usepackage{amsthm}
\usepackage{ulem}
\usepackage{indentfirst}
\usepackage{enumerate}
\usepackage{enumitem}
\usepackage{titlesec}

% Chinese numbering
\AddEnumerateCounter{\chinese}{\chinese}{}

\newcommand*\CJKmovesymbol[1]{\raise.35em\hbox{#1}}
\newcommand*\CJKmove{\punctstyle{plain}% do not modify the spacing between punctuations
	\let\CJKsymbol\CJKmovesymbol
	\let\CJKpunctsymbol\CJKsymbol}

% landscape
\usepackage{atbegshi}
\AtBeginShipout{
  \global\setbox\AtBeginShipoutBox\vbox{
    \special{pdf: put @thispage <</Rotate 90>>}
    \box\AtBeginShipoutBox
  }
}

% font
\defaultCJKfontfeatures{RawFeature={vertical:+vert}}
\makeatletter
\newcommand*{\shifttext}[2]{
  \settowidth{\@tempdima}{#2}
  \makebox[\@tempdima]{\hspace*{#1}#2}
}
\makeatother
\newcommand\ytza[1]{\shifttext{-3.5pt}{\ytzfz #1}}

% underline
%\renewcommand{\ULdepth}{7pt}
\renewcommand*{\uwave}{%
	\bgroup
	\markoverwith{%
		\lower5pt\hbox{\sixly\char58}%
	}%
	\ULon
}


%% font settings
\setCJKmainfont[BoldFont=SourceHanSerifSC-Bold.otf,ItalicFont=FZKai-Z03.TTF]{SourceHanSerifSC-Medium.otf}
\setCJKsansfont{FZKai-Z03}
\setCJKmonofont{FZKai-Z03}

\renewcommand{\thepage}{\Chinese{page}}

\usepackage{geometry}
\newgeometry{
  top=65pt, bottom=65pt, left=60pt, right=60pt,
  headsep=7pt, footskip=18pt
}
\savegeometry{mdGeo}
\loadgeometry{mdGeo}


\newcommand{\pageCN}{第\thepage 页}

\usepackage{fancyhdr}
\fancypagestyle{plain}{
    \fancyhf{}
%    \fancyhead[LE]{\quad\quad\quad\quad\small{\ziju{0.3}\leftmark}}
%    \fancyhead[RE]{\small\thepage\quad\quad\quad\quad\quad\quad}
%    \fancyfoot[LO]{\quad\quad\quad\quad\small{\ziju{0.3}\rightmark}}
%    \fancyfoot[RO]{\small\thepage\quad\quad\quad\quad\quad\quad\quad\quad}
    \renewcommand{\headrulewidth}{0bp}
}
\fancypagestyle{empty}{
    \fancyhf{}
    \renewcommand{\headrulewidth}{0bp}
}

% titles
\titleformat{\chapter}{}{}{0pt}{\raggedright\bfseries\LARGE\ziju{0.4}}
\titlespacing*{\chapter}{10pt}{0pt}{10pt}
\titleformat{\section}{}{}{0pt}{\raggedright\bfseries\Large\ziju{0.4}\hspace*{23pt}}
\titlespacing*{\section}{0pt}{8pt}{18pt}

% headers
\pagestyle{fancy}
\fancyhf{}
\fancyhead[LE]{\quad\quad\quad\quad\small{\ziju{0.3}\leftmark}}
\fancyhead[RE]{\small\thepage\quad\quad\quad\quad\quad\quad}
\fancyfoot[LO]{\quad\quad\quad\quad\small{\ziju{0.3}\rightmark}}
\fancyfoot[RO]{\small\thepage\quad\quad\quad\quad\quad\quad\quad\quad}
\renewcommand{\headrulewidth}{0bp} % 页眉线宽度


%% \RequirePackage{titletoc}

%% \titlecontents{chapter}[0pt]{\heiti\zihao{-4}}{\thecontentslabel\ }{}
%%               {\hspace{.5em}\titlerule*[4pt]{$\cdot$}\contentspage}
%%               \titlecontents{section}[2em]{\vspace{0.1\baselineskip}\songti\zihao{-4}}{\thecontentslabel\ }{}
%%                             {\hspace{.5em}\titlerule*[4pt]{$\cdot$}\contentspage}
%%                             \titlecontents{subsection}[4em]{\vspace{0.1\baselineskip}\songti\zihao{-4}}{\thecontentslabel\ }{}
%% {\hspace{.5em}\titlerule*[4pt]{$\cdot$}\contentspage}


\ctexset{
  today = big,
  punct = quanjiao, %% banjiao,
  autoindent = 25pt
}

\renewcommand{\chaptermark}[1]{%
	\markboth{#1}
	{\noexpand\firstsectiontitle}}
\renewcommand{\sectionmark}[1]{%
	\markright{#1}\gdef\firstsectiontitle{#1}}
%\renewcommand{\sectionmark}[1]{%
%	\markright{#1}%
%	\iffirstsectionmark
%	\gdef\firstsectiontitle{#1}%
%	\fi
%	\global\firstsectionmarkfalse}
%\newif\iffirstsectionmark
\def\firstsectiontitle{}


\title{\zihao{1}\textbf{微積溯源}}

\author{\normalsize 華蘅芳}
\date{\normalsize\today 版}


\begin{document}
%\maketitle
%\tableofcontents

\setcounter{chapter}{0}

% indent

\pagestyle{fancy}

\CJKmove

\large
\chapter {微积溯源序}

  \uwave{微積溯源}八卷,前四卷爲微分術,後四卷爲積分術,乃算學中最深之事也。余既與西士\uline{傅蘭雅}譯畢\uwave{代數術}二十五卷,更思求其進境,故又與傅君譯此書焉。先是咸豐年間,曾有\uline{海寗}\uline{李壬叔}與西士\uline{偉烈亞力}譯出代\uwave{微積拾級}一書,流播海内。余素與\uline{壬叔}相友,得讀其書,粗明微積二術之梗概。所以又譯此書者,蓋欲補其所略也。書中代數之式甚繁,校算不易,則\uline{劉君省菴}之力居多。

  今刻工已竣矣,故序之,曰:吾以爲古時之算法惟有加減而已。其乘與除乃因加減之不勝其繁,故更立二術以使之簡易也。開方之法,又所以濟除法之窮者也。蓋算學中自有加減乘除開方五法,而一切淺近易明之數,無不可通矣。惟人之心思智慮日出不窮,往往以能人之所不能者爲快。遇有窒礙難通之處,輒思立法以濟其窮。故有減其所不可減而正負之名不得不立矣;除其所不能除而寄母通分之法又不得不立矣。代數中種種記號之法皆出於不得已而立者也,惟每立一法必能使繁者爲簡,難者爲易,遲者爲速,而算學之境界藉此得更進一層。如是屢進不已而所立之法於是乎日多矣。微分積分者,蓋又因乘除開方之不勝其繁,且有窒礙難通之處,故更立此二術以濟其窮,又使簡易而速者也。試觀圓徑求周、真數求對等事,雖無微分積分之時,亦未嘗不可求,惟須乘除開方數十百次。其難有不可言喻者,不如用微積之法理明而數捷也。然則謂加減乘除開方代數之外者,更有二術焉,一曰微分,一曰積分可也。其積分術爲微分之還原,猶之開平方爲自乘之還原、除法爲乘之還原、減法爲加之還原也。然加與乘其原無不可還,而微分之原有可還有不可還,是猶算式中有不可開之方耳,又何怪焉。如必曰加減乘除開方已足供吾之用矣,何必更究其精?是舍舟車之便利而必欲負重遠行也。其用力多而成功少,蓋不待智者而辨矣。同治十三年九月十八日,\uline{金匱}\uline{華蘅芳}序。

\chapter {微積溯源卷一}
\setcounter{page}{1}
\section {論變數与函數之變比例}
\thispagestyle{fancy}
\begin{enumerate} [label={第\chinese*款}]
	\item 用代數以解任何曲線,其中每有幾種數,其大小恒有定率者,如橢圓之長徑、抛物線之通徑、雙曲線之屬徑之類是也。\\
	又每有幾種數可有任若干相配之同數,其大小恒不能有定率者,如曲線任一點之縱橫線是也。\\
	數既有此兩種分別,則每種須有一總名以賅之,故名其有定之數曰常數,無定之數曰變數。\\
	凡常數之同數不能增亦不能損。\\
	凡變數之同數,能變爲大,亦能變爲小。故其從此同數變至彼同數之時,必歷彼此二數閒最小最微之各分數。\\
	如平圓之半徑爲常數,而其任一段之弧或弧之弦矢切割各線、及各線與弧所成之面,皆謂之變數。\\
	橢圓之長徑短徑皆爲常數,而其曲線之任一段或曲線上任一點之縱橫線,並其形内形外所能作之任何線或面或角,皆謂之變數。\\
	抛物線之通徑爲常數,而其曲線之任一段或任一點之縱橫線、或弧與縱橫線所成之面,皆謂之變數。他種曲線亦然。\\
	凡常數,恒以甲乙丙丁等字代之。凡變數,恒以天地人等字代之。
	\item 若有彼此二數皆爲變數,此數變而彼數因此數變而亦變者,則彼數爲此數之函數。\\
	如平圓之八線皆爲弧之函數。若反求之,亦可以弧爲八線之函數。\\
	又如重學中令物體前行之力,與其物所行之路,皆爲時刻之函數。\\
	如有式\CJKmoveback{\equa{$\displaystyle \text{地}=\frac{\text{甲}\tang\text{天}}{\text{甲}\perp\text{天}}$}}\CJKmove,此式中甲爲常數,天爲自主之變數,地爲天之函數。故地之同數能以天與甲明之。\\
	如有式\CJKmoveback{\equa{$\displaystyle \textit{天}=\frac{\text{地}\perp\text{一}}{\text{甲}(\text{地}\tang\text{一})}$}}\CJKmove,此式中甲與一皆爲常數,地爲自主之變數,天爲地之函數,故天之同數可以地與甲及一明之。\\
	如有式\CJKmoveback{\equa{$\textit{戌}=\textit{甲}\perp\textit{乙天}\perp\textit{丙天}^{\textit{二}}$}}\CJKmove 或\CJKmoveback{\equa{$\displaystyle\textit{戌}=\sqrt{\textit{甲}^{\textit{二}}\perp\textit{乙天}\perp\textit{天}^{\textit{二}}}$}}\CJKmove 或\CJKmoveback{\equa{$\displaystyle\textit{戌}=\frac{\textit{丙}\perp\textit{天}^{\textit{二}}}{\textit{甲}\perp\sqrt{\textit{乙天}}}$}}\CJKmove,其甲乙丙爲常數,天爲自變之數,而戌皆爲天之函數。\\
	凡函數之中,可以有數箇自主之變數。\\
	如有式\CJKmoveback{\equa{$\textit{戌}=\textit{甲天}^{\textit{二}}\perp\textit{乙天地}\perp\textit{丙地}^{\textit{二}}$}}\CJKmove 則天與地皆爲自主之變數,戌爲天地兩變數之函數。\\
	凡變數之函數,其形雖有多種,然每可化之,使不外乎以下數類\CJKmoveback{\equa{$\textit{天}^{\text{卯}}$}}\CJKmove、\CJKmoveback{\equa{$\textit{甲}^{\textit{天}}$、$\textit{正弦天}$、$\textit{餘弦天}$}}\CJKmove 等類是也。\\
	凡函數爲\CJKmoveback{\equa{$\textit{天}^{\text{卯}}$}}\CJKmove 之類,其指數爲常數,則可從天之卯方,用代數之常法化之。\\
	而以有窮之項明其函數之同數。故謂之代數函數,亦謂之常函數。\\
\end{enumerate}

\end{document}
